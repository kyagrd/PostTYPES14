\vspace*{7.5ex}
\begin{lstlisting}[
	caption={An embedding algorithm in Haskell
		% for the constructors of
		% regular datatypes used with \lstinline{msfcata0}
		% transcribed in Haskell.
		\label{lst:embed}}]
data Sig = P | M -- %\textcomment{$P$ and $M$ stands for $+$ and $-$ %}

data Ty = TV Var | All Var Ty | Ty :-> Ty
        | Unit | Ty :*: Ty -- %\textcomment{These can be encoded too but%}
        | Void | Ty :+: Ty -- %\textcomment{for convenience of presentation.%}
        
data Tm = Fn (Tm -> Tm) | Tm :$ Tm -- %\textcomment{term representation using HOAS%}
        | Cunit | Tm :* Tm | Fst Tm | Snd Tm
        | L Tm | R Tm | CaseLR Tm (Tm -> Tm) (Tm -> Tm)
	| Lift Tm Tm | Phi -- %\textcomment{Added constants for%} lift %\textcomment{and%} phi.

type Var = Int -- %\textcomment{Give some suitable type for variable.%}

flipSig :: Sig -> Sig
flipSig M = P
flipSig P = M

-- %\textcomment{For a base structure $F$ defined as ~ \textbf{data}%} F r = C t1 ... tn | ... ,
-- %\textcomment{the embedding of the constructors of %} Rec0 F %\textcomment{ has the form of %}
-- %\textcomment{$c = \inR\;(\lambda \varphi.\varphi~\eta^{-1}~(C\;(\textit{rEm}\;t_1)\;\cdots\;(\textit{rEm}\;t_n)))))$ where \textit{rEm} is defined as below:%}
rEm :: Sig -> Var -> Ty -> Tm -> Tm
rEm _ _ Unit      = id -- %\textcomment{or %} const Cunit
rEm _ _ Void      = id
rEm _ r (TV x) | r/=x = id -- %\textcomment{Ignore variables other than the recursive one.%}
rEm P _ (TV _)   = Lift Phi -- %\textcomment{Apply%} lift phi %\textcomment{in positive occurrence%}
rEm M _ (TV _)   = L       -- %\textcomment{and apply $\inL$ in negative occurrence.%}
rEm p r (a :-> b) = \f -> Fn (\x -> rEm p r b (f :$ rEm p' r a x))
                  where p' = flipSig p
rEm p r (a :*: b) = \x -> rEm p r a x :* rEm p r b x
rEm p r (a :+: b) = \x -> CaseLR x (rEm p r a) (rEm p r b)
rEm p r (All x b) | r/=x = rEm p r b
rEm _ _ (All _ _) = error "should have been alpha renamed"
\end{lstlisting}

\newpage
\appendix
\section{Appendix:
	a type-directed embedding algorithm for the constructors of
	regular datatypes used with \lstinline{msfcata0}} \label{sec:appendix}

In \S\ref{sec:theory:embed}, we embedded the type constructors of
the untyped HOAS, which is a mixed-variant datatype with both positive
and negative occurrences, as annotated by $+$ and $-$
in $\textit{App}~r^{+}~r^{+}$ and $\textit{Abs}\;(r^{-} \to r^{+})$.
The HOAS example discussed in \S\ref{sec:theory:embed} has these
recursive occurrences either at topmost level, as in $\textit{App}~r^{+}~r^{+}$
occurring twice positively, or on both sides of the arrow type at topmost level,
as in $\textit{Abs}\;(r^{-} \to r^{+})$ occurring negatively
on the left-hand side and positively on the right-hand side.
Positive and negative occurrences are embedded differently --
recall that we used \lstinline{lift phi} for positive occurrences and $\inL$ for
negative occurrences (see p\pageref{align:embed} in \S\ref{sec:theory:embed}).

In general, recursive occurrences may occur more deeply inside
the type structure. For example, consider \lstinline{Rec0 F} where
\lstinline{data F r =  C}~$((r^{+} \to r^{-}) \to r^{+})$.
The leftmost occurrence of $r$ in the definition of $F$ is positive
because it is on the left hand side of the arrow at negative position
(negative of negative considered positive). Other type structures such as sums,
products, and universal quantifications do not have affect on the sign of
recursive occurrences in its subcomponents. That is, the subcomponents
maintain the same sign for recursive occurrences as their outermost position.

In Listing~\ref{lst:embed}, we describe an algorithm implementing the idea
discussed in the previous two paragraphs using Haskell. This algorithm is
type-directed, that is, it analyzes the given base structure $F$ to derive
the embeddings for the constructors of \lstinline{Rec0 F}. Here, we only
consider regular datatypes. By convention, the recursive argument $r$ always
comes at the last. For instance, the base structure for lists
\lstinline{data L a r = ...} where we take its fixpoint as \lstinline{Mu0(L a)}
for the list datatype. Therefore without loss of generality, we assume
that the base structures are defined as \lstinline{data F r = ...}.
Since our target calculus is polymorphic, we need variables (\textit{Var})
and universal quantification (\textit{All}) to represent types (\textit{Ty}).
We have sums (\lstinline{:+:}) and products (\lstinline{:*:}) and their
identities \lstinline{Unit} and \lstinline{Void} because base structures are
defined as sums of products of types. We can inline the embeddings of recursive
types of the form \lstinline{Rec0 G} occurring in the definition of $F$,
provided that $G$ is defined prior to $F$, because we already know
the embedding of \lstinline{Rec0} (see p\pageref{sec:theory:embed} in
\S\ref{sec:theory:embed}) and $G$ can also be embedded into System \Fw\ as
sums of products. Therefore, it suffice for the embedding function \textit{rEm}
in Listing~\ref{lst:embed} to analyze type structures (\textit{Ty})
in order to generate the embedded terms (\textit{Tm}).

We think it would be possible to prove the equational properties of
this type-directed embedding using interpolation, as used in the paper
by Matthes \cite{Matthes01a}. In fact, his paper has been a hint to derive
our algorithm in Listing~\ref{lst:embed}. Although we have only demonstrated
the algorithm for regular datatypes, we do not expect difficulties
in generalizing this algorithm to include non-regular datatypes, except for
truly nested datatypes
(e.g., \lstinline{data Bush a = BNil | BCons Bush (Bush a)}).
Embeddings for truly nested datatypes are going to be trickier than
the embeddings for the other datatypes because truly nested datatype
are indexed by their own types.

