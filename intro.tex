\section{Introduction}\label{sec:intro}

Inspired by Mendler \cite{Mendler87}, Uustalu, Matthes, and others
\cite{UusVen99,UusVen00,AbeMatUus03,AbeMatUus05,AbeMat04} have studied
and generalized Mendler's formulation of primitive recursion. They coined
the term \emph{Mendler style} for this new way of formulating recursion schemes
and called the previous prevalent approach \emph{conventional style} (\eg,
the Squiggol school and structural/lexicographic termination checking
as used in proof assistants). Advantages of the Mendler style, in contrast to
the conventional style, include:
\begin{itemize}
\item Admitting arbitrary recursive datatype definitions
	(including mixed-variant ones),
\item Succinct and intuitive usability of recursion schemes
	(code looks like general recursion),
\item Uniformity of recursion scheme definition across all datatypes
	(including indexed ones),
\item Type-based termination
	(not relying on any external theories other than type checking).
\end{itemize}
Primary focus of this work is on the first advantage, but other advantages
are discussed and demonstrated by examples throughout this paper.

Early work \cite{UusVen99,UusVen00,AbeMatUus03,AbeMatUus05,AbeMat04} on
the Mendler style noticed the first advantage but focused on examples
using positive datatypes. Recently, Ahn and Sheard \cite{AhnShe11}
discovered a Mendler-style recursion scheme \msfit{} over mixed-variant datatypes
(inspired by earlier work \cite{MeiHut95,FegShe96,bgb} in the conventional setting).
Using \msfit{}, they demonstrated a HOAS formatting example (\S\ref{sec:mendler:sf}) over
a non-indexed HOAS. This example was adopted from earlier work \cite{FegShe96,bgb} in
the conventional style. In this paper we demonstrate that \msfit{} is useful over
indexed datatypes as well (\S\ref{sec:HOASeval}).

Ahn and Sheard \cite{AhnShe11} gave a semi-formal termination proof by
embedding \msfit{} into subset of Haskell that is believed to be a subset of System~\Fw.
Here, we investigate its properties in a more rigorous theoretical
setting (\S\ref{sec:theory}).

In this paper, we give an introduction to the Mendler style by reviewing
Mender-style iteration (\MIt{}) and iteration with syntactic inverses (\msfit{})
over regular (\ie, non-indexed) datatypes. Next, we demonstrate the usefulness of
the Mendler-style recursion scheme \msfit{} over
indexed and mixed-variant datatypes (\S\ref{sec:HOASeval}).
We report our novel discovery that
a type-preserving evaluator for a simply-typed HOAS can be defined
using \lstinline{msfcata}; this indicates that a simply-typed HOAS evaluator
can be embedded in System~\Fw\ with its correct-by-construction proof of
type-preservation and strong normalization. We do just that -- embedding \msfit{}
into System \Fw\ (\S\ref{sec:theory:embed}). We also show that
the equational properties of \msfit{} are faithfully transferred to its
\Fw-embedding (\S\ref{sec:theory:eqlam}, \S\ref{sec:theory:eqapp}).
Moreover, we discuss the relationship between ordinary fixpoints and
the inverse-augmented fixpoints used in \msfit{} (\S\ref{sec:ongoing}),
and introduce two new recursion schemes over mixed-variant datatypes
(\S\ref{sec:ongoing}).

Our contributions can be listed as follows:
\begin{enumerate}
\item Demonstrating the usefulness of the Mendler style
	over indexed and mixed-variant datatypes,
\item Writing a simply-typed HOAS evaluator using \msfit{},
	whose type-preservation and termination properties are guaranteed
	simply by type checking (\S\ref{sec:HOASeval}),
\item Clarifying the relation between
	fixpoints of \MIt{} and fixpoints of \msfit{} (\S\ref{sec:murec}),
\item Embedding \msfit{} into System \Fw\ (\S\ref{sec:theory:embed}),
\item Proving equational properties regarding the \Fw-embedding of \msfit{} 
	(\S\ref{sec:theory:eqlam}, \S\ref{sec:theory:eqapp}),
\item Formulating the Mendler-style primitive recursion with a size-index %%% sized index
	(\S\ref{sec:ongoing:mprsi}), and
\item Formulating another Mendler-style iteration with syntactic inverses
	\textit{\`a la} PHOAS (\S\ref{sec:ongoing:mphit}).
\end{enumerate}
