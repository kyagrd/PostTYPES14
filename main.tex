%This is a template for producing LIPIcs articles.
%See lipics-manual.pdf for further information.

\documentclass[a4paper,UKenglish]{lipics}
  %for A4 paper format use option "a4paper", for US-letter use option "letterpaper"
  %for british hyphenation rules use option "UKenglish", for american hyphenation rules use option "USenglish"
 % for section-numbered lemmas etc., use "numberwithinsect"
 
\usepackage{microtype}%if unwanted, comment out or use option "draft"

%\graphicspath{{./graphics/}}%helpful if your graphic files are in another directory

\bibliographystyle{plain}% the recommended bibstyle

% Author macros %%%%%%%%%%%%%%%%%%%%%%%%%%%%%%%%%%%%%%%%%%%%%%%%
\title{Mendler-style Recursion Schemes for Mixed-Variant Datatypes
%%\footnote{This work was partially supported by TODO list grant etc.}
	}
%% \titlerunning{A Sample LIPIcs Article} %optional, in case that the title is too long; the running title should fit into the top page column

\author[1]{Ki Yung Ahn}
\author[1]{Tim Sheard}
\author[2]{Marcelo Fiore}
\affil[1]{Department of Computer Science, Portland State University\\
  Oregon, USA\\
  \texttt{\{kya,sheard\}@cs.pdx.edu}}
\affil[2]{Computer Laboratory, University of Cambridge\\
  Cambridge, UK\\
  \texttt{Marcelo.Fiore@cl.cam.ac.uk}}
\authorrunning{KY. Ahn, T. Sheard and M. Fiore} %mandatory. First: Use abbreviated first/middle names. Second (only in severe cases): Use first author plus 'et. al.'

\Copyright{Ki Yung Ahn, Tim Sheard and Marcelo Fiore}%mandatory. LIPIcs license is "CC-BY";  http://creativecommons.org/licenses/by/3.0/

\subjclass{
D.3.3 [Programming Languages]: Language Constructs and Features
--- Data types and structures, Polymorphism, and Recursion;
F.3.3 [Logics and Meanings of Programs]: Studies of Program Constructs
-- Functional constructs, Program and recursion schemes, and Type structure;
F.4.1 [Mathematical Logic and Formal Systems]: Mathematical Logic
--- Lambda calculus and related systems
}
\keywords{
Mendler-style recursion, higher-order abstract syntax, HOAS,
mixed-variant datatypes, negative datatypes, termination, normalization}
% mandatory: Please provide 1-5 keywords
%%%%%%%%%%%%%%%%%%%%%%%%%%%%%%%%%%%%%%%%%%%%%%%%%%%%%%%%%

%Editor-only macros (do not touch as author)%%%%%%%%%%%%%%%%%%%%%%%%%%%%%%%%%%%
\serieslogo{}%please provide filename (without suffix)
\volumeinfo%(easychair interface)
  {Billy Editor, Bill Editors}% editors
  {2}% number of editors: 1, 2, ....
  {Conference title on which this volume is based on}% event
  {1}% volume
  {1}% issue
  {1}% starting page number
\EventShortName{}
\DOI{10.4230/LIPIcs.xxx.yyy.p}% to be completed by the volume editor
%%%%%%%%%%%%%%%%%%%%%%%%%%%%%%%%%%%%%%%%%%%%%%%%%%%%%%%%%

\newcommand{\cf}[0]{{cf.}}
\newcommand{\eg}[0]{{e.g.}}
\newcommand{\ie}[0]{{i.e.}}
\newcommand{\aka}[0]{{a.k.a.}}

\newcommand{\F}[0]{{\ensuremath{\mathsf{\uppercase{F}}}}}
\newcommand{\Fw}[0]{{\ensuremath{\mathsf{\uppercase{F}}_{\!\omega}}}}
\newcommand{\Fixw}[0]{{\ensuremath{\mathsf{\uppercase{F}\lowercase{ix}}_{\omega}}}}
\newcommand{\Fwhat}[0]{{\ensuremath{\mathsf{\uppercase{F}}_{\!\omega}\!\!\char`\^}}}

\newcommand{\MIt}[1]{\ensuremath{\textbf{\textit{mit}}_{#1}}}
\newcommand{\MPr}[1]{\ensuremath{\textbf{\textit{mpr}}_{#1}}}
\newcommand{\mpr}[1]{\ensuremath{\textbf{\textit{mpr}}_{#1}}}
\newcommand{\MsfIt}[1]{\ensuremath{\textbf{\textit{msfit}}_{#1}}}
\newcommand{\msfit}[1]{\ensuremath{\textbf{\textit{msfit}}_{#1}}}
\newcommand{\mprsi}[1]{\ensuremath{\textbf{\textit{mprsi}}_{#1}}}
\newcommand{\mcvpr}[1]{\ensuremath{\textbf{\textit{mcvpr}}_{#1}}}
\newcommand{\mphit}[1]{\ensuremath{\textbf{\textit{mphit}}_{#1}\,}}
\newcommand{\mphpr}[1]{\ensuremath{\textbf{\textit{mphpr}}_{#1}}}
\newcommand{\mphcv}[1]{\ensuremath{\textbf{\textit{mphcv}}_{#1}}}
\newcommand{\lift}[0]{\textit{lift}}
\newcommand{\In}[1]{\ensuremath{\textbf{\textit{In}}_{#1}}}
\newcommand{\InC}[1]{\ensuremath{\textbf{\textit{In}}^{_{\textbf{\textit{C}}}}_{#1}}}
\newcommand{\unInC}[1]{\ensuremath{\textbf{\textit{In}}^{_{\textbf{\textit{C}}}\;-\!1}_{#1}}}
\newcommand{\hatIn}[1]{\ensuremath{\textbf{\textit{I\^n}}_{#1}}}
\newcommand{\hatunIn}[1]{\ensuremath{\textbf{\textit{I\^n}}^{-1}_{#1}}}
\newcommand{\inL}[0]{\textit{in}_{L}}
\newcommand{\inR}[0]{\textit{in}_{R}}

\newcommand{\textcomment}[1]{\rm\color{blue}#1}

%% \setlength{\thickmuskip}{0mu}
%% \setlength{\medmuskip}{0mu}
%% \setlength{\thinmuskip}{0mu}
\lstset{captionpos=t,
	float,
	keepspaces=false,
	abovecaptionskip=-\medskipamount,
	language=Haskell,
 	basicstyle=\it,
 	keywordstyle=\rm\bfseries,
	commentstyle=\color{blue},
	escapechar={\%},
	stringstyle=\rm\ttfamily,
	deletekeywords={Bool,Int,Integer,String,show,const,id,length,List},
 	literate=
 		{forall}{{$\forall$}}1
 		{:}{{$:$}}1
 		{:->}{{$:\to$}}2
 		{:+:}{{$\,:\!\!+\!\!:\,$}}2
		{:$}{{$:${\rm\textdollar}}}1
 		{:*}{{$\,:\!\!\times\,$}}2
 		{:*:}{{$\,:\!\!\times\!\!:\,$}}2
 		{;}{{$;$}}1
 		{::}{{$::$}}1
 		{.->}{{\!.$\to$\,}}2
 		{->}{{$\to$}}2
 		{=}{{$=$}}1
 		{=>}{{$\Rightarrow$}}2
 		{*}{{$*$}}1
		{/=}{{$\neq$}}1
 		{\\}{{$\lambda$}}1
 		{eta}{{$\eta$}}1
		{Phi0}{{Phi$_{*}$}}3
		{Phi1}{{Phi$_{*\to*}$}}5
		{PhiC_0}{{Phi$^{\,C}_{*}$}}3
		{PhiC_1}{{Phi$^{\,C}_{*\to*}$}}5
		{Phi0'}{{Phi$_{*}^{\phantom{.}\prime}$}}3
		{Phi1'}{{Phi$_{*\to*}^{\phantom{.}\prime}$}}5
		{Mu_0}{{$\hat\mu_{*}$}}1
		{Mu0}{{$\mu_{*}$}}1
		{In_0}{{\hatIn{*}}}2
		{unIn_0}{{\hatunIn{*}}}3
		{unIn_1}{{\hatunIn{*\to*}}}4
		{In0}{{\In{*}}}2
		{out0}{{out$_{*}$}}3
		{Mu_1}{{$\hat\mu_{*\to*}$}}3
		{Mu1}{{$\mu_{*\to*}$}}3
		{In_1}{{\hatIn{*\to*}}}4
		{In1}{{\In{*\to*}}}4
		{out1}{{out$_{*\to*}$}}5
		{MuC_0}{{$\mu^{c}_{*}$}}1
		{MuC_1}{{$\mu^{c}_{*\to*}$}}3
		{InC_0}{{\InC{*}}}2
		{InC_1}{{\InC{*\to*}}}4
		{unInC_0}{{\unInC{*}}}4
		{unInC_1}{{\unInC{*\to*}}}4
		{Rec0}{{$\mu_{*}^\prime$}}1
		{Roll0}{{$\In{*}^\prime$}}2
		{unRoll0}{{unIn$_{*}^\prime$}}4
		{Var_0}{{Var$_{*}$}}3
		{Inverse0}{{Inverse$_{*}$}}6
		{Rec1}{{$\mu_{*\to*}^\prime$}}3
		{Roll1}{{$\In{*\to*}^\prime$}}4
		{unRoll1}{{unIn$_{*\to*}^\prime$}}6
		{Var_1}{{Var$_{*\to*}$}}5
		{Inverse1}{{Inverse$_{*\to*}$}}8
		{mphit}{{$\mphit{}$}}5
		{mphit0}{{$\mphit{*}$}}6
		{mphit1}{{$\mphit{*\to*}$}}8
		{var}{{\textit{var}}}2
 		{phi}{{$\varphi$}}1
		{r_}{{$r_{\!\frac{\;\,}{\,\;}}$}}2
		{id}{{\textit{id}}}2
 		{t1}{{$t_1$}}1
 		{t1}{{$t_1$}}1
 		{tn}{{$t_n$}}1
 		{t2}{{$t_2$}}1
 		{e1}{{$e_1$}}1
 		{e2}{{$e_2$}}1
 		{p'}{{$p^\prime$}}1
 		{e1'}{{$e_1^\prime$}}1
 		{e2'}{{$e_2^\prime$}}1
		{pr}{{pr}}2
		{times}{{times}}4
		{mpr}{{mpr}}3
		{mpr0}{{\mpr{*}}}4
		{mpr1}{{\mpr{* \to *}}}6
		{mprim}{{\mpr{}}}3
		{mprim0}{{\mpr{*}}}4
		{mprim1}{{\mpr{* \to *}}}6
		{mprsi}{{\mprsi{}}}5
		{mprsi1}{{\mprsi{* \to *}}}7
		{mcvpr}{{\mcvpr{}}}5
		{mcvpr0}{{\mcvpr{*}}}6
		{mcvpr1}{{\mcvpr{* \to *}}}8
		{mphpr}{{\mphpr{}}}5
		{mphpr0}{{\mphpr{*}}}6
		{mphpr1}{{\mphpr{* \to *}}}8
		{mphcv}{{\mphcv{}}}5
		{mphcv0}{{\mphcv{*}}}6
		{mphcv1}{{\mphcv{* \to *}}}8
		{uncast}{{uncast}}6
		{out}{{out}}2
		{cast}{{cast}}3
		{call}{{call}}3
		{fac}{{fac}}2
		{zero}{{zero}}3
		{succ}{{succ}}3
		{pred}{{pred}}3
		{All}{{All}}2
		{Lift}{{Lift}}3
		{lift}{{lift}}2
		{Unit}{{Unit}}3
		{Void}{{Void}}3
		{Sig}{{Sig}}3
		{flipSig}{{flipSig}}5
		{factorial}{{factorial}}6
		{unVal}{{un\!Val}}4
		{mcata}{{\MIt{}}}3
		{mcata0}{{\MIt{*}}}3
		{mcata1}{{\MIt{* \to *}}}5
		{msfcata}{{\msfit{}}}4
		{msfcata'}{{\msfit{}$^\prime$}}5
		{msfcata0}{{\msfit{*}}}5
		{msfcata1}{{\msfit{* \to *}}}7
 		{+}{{$+$}}1
		{++}{{$+\!\!\!+$}}2
 		{(}{{$($}}1
 		{)}{{$)$}}1
 		{[}{{$[$}}1
 		{]}{{$]$}}1
 		{\{}{{$\{$}}1
 		{\}}{{$\}$}}1
 		{|}{{$\mid$}}1
		{Exp}{{Exp}}3
		{Expr}{{Expr}}4
		{Exp_u}{{Exp$_u$}}3
		{ExpF}{{ExpF}}4
		{ExprF}{{ExprF}}5
		{ExpF_u}{{ExpF$_u$}}4
		{App_u}{{App$_u$}}4
		{Lam_u}{{Lam$_u$}}4
		{V_u}{{V\!$_u$}}1
		{Val}{{Val}}3
		{VFun_u}{{VFun$_u$}}4
		{Expr'}{{Expr$\,^\prime$}}5
		{Exp'}{{Exp$\,^\prime$}}4
		{show'}{{show$\,^\prime$}}4
		{desug}{{desug}}4
		{desugExp}{{desugExp}}8
		{showExp}{{showExp}}7
		{String}{String}5
		{Bool}{{Bool}}4
		{List}{{List}}3
		{Int}{{Int}}3
		{Fst}{{Fst}}3
		{Snd}{{Snd}}3
		{Cunit}{{Cunit}}4
		{inv}{{inv}}3
		{nil}{{nil}}2
		{len}{{len}}3
		{length}{{length}}4
		{cons}{{cons}}3
		{LET}{LET}3
		{LIT}{LIT}3
		{ADD}{ADD}3
		{cf}{cf}1
		{eLet}{eLet}3
		{eLit}{eLit}3
		{eAdd}{eAdd}3
		{const}{const}4
		{constfold}{{constfold}}6
		{ev}{{ev}}2
		{val}{{val}}2
		{eval}{{eval}}3
		{veval}{{veval}}4
		{vs}{{vs}}2
		{vars}{{vars}}3
		{lam}{{lam}}3
		{`}{{$^\backprime$}}1
		{app}{{app}}3
		{`app`}{{$^\backprime\textit{app}^\backprime$}}4
		{Id}{{K}}1
		{MkId}{{$\eta$}}2
		{unId}{{$\eta^{-1}$}}2
		{exp2expr}{{exp2expr}}7
		{expr2exp'}{{expr2exp$^\prime$}}7
		{exp'2expr}{{exp$^\prime$2expr}}7
		{p2r}{{p2r}}2
}


\begin{document}

\maketitle

\begin{abstract}
Some concepts, such as Higher-Order Abstract Syntax (HOAS),
are most naturally expressed by \emph{mixed-variant datatypes}
(\aka\ negative (recursive) datatypes). Unfortunately,
mixed-variant datatypes are often outlawed in formal reasoning systems
based on the Curry--Howard correspondence (\eg, Coq, Agda), because
the conventional recursion schemes (or induction principles) supported in
such systems cannot guarantee termination for mixed-variant datatypes.

There is an alternative style of formulating recursion schemes,
known as the Mendler style, that can guarantee termination for
arbitrary datatypes. Ahn and Sheard \cite{AhnShe11} formulated
a Mendler-style recursion scheme (\msfit{}), and provided
examples involving regular (\ie, non-indexed) mixed-variant datatypes
(\eg, untyped $\lambda$-calculus in HOAS). Their examples demonstrate
an advantage of the Mendler style -- a termination guarantee for
arbitrary datatypes, including mixed-variant ones. They proved
termination of the examples via an embedding into System~\Fw.

Another advantage of the Mendler style is that recursion schemes
naturally extend to non-regular (\ie, indexed) datatypes. In this paper,
we provide another example: a type-preserving evaluator for a simply-typed HOAS
defined as a type-indexed mixed-variant datatype. This example demonstrates
both advantages of the Mendler style.

This example illustrates a novel discovery that the simply-typed HOAS evaluator
is expressible within System~\Fw. To our knowledge, this is the first example of
a simply-typed HOAS evaluator (without translation through first-order syntax)
that is equipped with correct-by-construction proofs (in the Curry--Howard
sense) of both type-preservation and normalization. We also develop further
theoretical discussions on the \Fw-embedding of \msfit{}
and introduce further studies on two new recursion schemes
(\mprsi{} and \mphit{}), which are also useful for mixed-variant datatypes.
We hope our work motivates future design of
logical reasoning systems that support a wider range of datatypes,
including mixed-variant ones.
\end{abstract}

\section{Introduction}\label{sec:intro}

Inspired by Mendler \cite{Mendler87}, Uustalu, Matthes, and others
\cite{UusVen99,UusVen00,AbeMatUus03,AbeMatUus05,AbeMat04} have studied
and generalized Mendler's formulation of primitive recursion. They coined
the term \emph{Mendler style} for this new way of formulating recursion schemes
and called the previous prevalent approach \emph{conventional style} (\eg,
the Squiggol school and structural/lexicographic termination checking
as used in proof assistants). Advantages of the Mendler style, in contrast to
the conventional style, include:
\begin{itemize}
\item Admitting arbitrary recursive datatype definitions
	(including mixed-variant ones),
\item Succinct and intuitive usability of recursion schemes
	(code looks like general recursion),
\item Uniformity of recursion scheme definition across all datatypes
	(including indexed ones),
\item Type-based termination
	(not relying on any external theories other than type checking).
\end{itemize}
Primary focus of this work is on the first advantage, but other advantages
are discussed and demonstrated by examples throughout this paper.

Early work \cite{UusVen99,UusVen00,AbeMatUus03,AbeMatUus05,AbeMat04} on
the Mendler style noticed the first advantage but focused on examples
using positive datatypes. Recently, Ahn and Sheard \cite{AhnShe11}
discovered a Mendler-style recursion scheme \msfit{} over mixed-variant datatypes
(inspired by earlier work \cite{MeiHut95,FegShe96,bgb} in the conventional setting).
Using \msfit{}, they demonstrated a HOAS formatting example (\S\ref{sec:mendler:sf}) over
a non-indexed HOAS. This example was adopted from earlier work \cite{FegShe96,bgb} in
the conventional style. In this paper we demonstrate that \msfit{} is useful over
indexed datatypes as well (\S\ref{sec:HOASeval}).

Ahn and Sheard \cite{AhnShe11} gave a semi-formal termination proof by
embedding \msfit{} into subset of Haskell that is believed to be a subset of System~\Fw.
Here, we investigate its properties in a more rigorous theoretical
setting (\S\ref{sec:theory}).

In this paper, we give an introduction to the Mendler style by reviewing
Mender-style iteration (\MIt{}) and iteration with syntactic inverses (\msfit{})
over regular (\ie, non-indexed) datatypes. Next, we demonstrate the usefulness of
the Mendler-style recursion scheme \msfit{} over
indexed and mixed-variant datatypes (\S\ref{sec:HOASeval}).
We report our novel discovery that
a type-preserving evaluator for a simply-typed HOAS can be defined
using \lstinline{msfcata}; this indicates that a simply-typed HOAS evaluator
can be embedded in System~\Fw\ with its correct-by-construction proof of
type-preservation and strong normalization. We do just that -- embedding \msfit{}
into System \Fw\ (\S\ref{sec:theory:embed}). We also show that
the equational properties of \msfit{} are faithfully transferred to its
\Fw-embedding (\S\ref{sec:theory:eqlam}, \S\ref{sec:theory:eqapp}).
Moreover, we discuss the relationship between ordinary fixpoints and
the inverse-augmented fixpoints used in \msfit{} (\S\ref{sec:ongoing}),
and introduce two new recursion schemes over mixed-variant datatypes
(\S\ref{sec:ongoing}).

Our contributions can be listed as follows:
\begin{enumerate}
\item Demonstrating the usefulness of the Mendler style
	over indexed and mixed-variant datatypes,
\item Writing a simply-typed HOAS evaluator using \msfit{},
	whose type-preservation and termination properties are guaranteed
	simply by type checking (\S\ref{sec:HOASeval}),
\item Clarifying the relation between
	fixpoints of \MIt{} and fixpoints of \msfit{} (\S\ref{sec:murec}),
\item Embedding \msfit{} into System \Fw\ (\S\ref{sec:theory:embed}),
\item Proving equational properties regarding the \Fw-embedding of \msfit{} 
	(\S\ref{sec:theory:eqlam}, \S\ref{sec:theory:eqapp}),
\item Formulating the Mendler-style primitive recursion with a size-index %%% sized index
	(\S\ref{sec:ongoing:mprsi}), and
\item Formulating another Mendler-style iteration with syntactic inverses
	\textit{\`a la} PHOAS (\S\ref{sec:ongoing:mphit}).
\end{enumerate}

\section{Mendler-style recursion schemes}
\label{sec:mendler}
In this section, we introduce basic concepts of two Mendler-style recursion
schemes: iteration (\lstinline{mcata}\,) and iteration with syntactic inverses
(\lstinline{msfcata}\,). Further details on Mendler-style recursion schemes,
including these two and more,
can be found in \cite{AhnShe11,AbeMatUus05,UusVen00,AbeMat04}.

\begin{figure}
\begin{lstlisting}[
	caption={Mendler-style iteration (\MIt{}) and
		Mendler-style iteration with syntactic inverses (\msfit{})
		at kind $*$ and $*\to *$ transcribed in Haskell
		\label{lst:reccomb}}]
data Mu0   (f::(* -> *))               = In0   (f (Mu0   f)  )
data Mu1 (f::(* -> *) -> (* -> *)) i = In1 (f (Mu1 f) i)

type a .-> b = forall i. a i -> b i
                      -- %\textcomment{call%}
type Phi0   f a = forall r. (r -> a) -> (f r -> a)
type Phi1 f a = forall r. (r .-> a) -> (f r .-> a)

mcata0   :: Phi0   f a ->  Mu0  f    -> a
mcata1 :: Phi1 f a ->  Mu1 f .-> a i

mcata0   phi (In0   x) = phi (mcata0   phi) x
mcata1 phi (In1 x) = phi (mcata1 phi) x
-- ------------------------------------------------------------
data Rec0   f a   = Roll0   (f (Rec0 f a)    ) | Inverse0   a
data Rec1 f a i = Roll1 (f (Rec1 f a) i) | Inverse1 (a i)
                      -- %\textcomment{inverse%}      -- %\textcomment{call%}
type Phi0'   f a = forall r. (a -> r a) -> (r a -> a) -> f (r a) -> a
type Phi1' f a = forall r. (a .-> r a) -> (r a .-> a) -> f (r a) .-> a

msfcata0   :: Phi0'   f a -> (forall a. Rec0   f a  ) -> a
msfcata1 :: Phi1' f a -> (forall a. Rec1 f a i) -> a i

msfcata0  phi r = msfcata phi r where
  msfcata phi (Roll0   x)      = phi Inverse0  (msfcata phi)  x
  msfcata phi (Inverse0 z)    = z

msfcata1  phi r = msfcata phi r  where
  msfcata  :: Phi1' f a -> Rec1 f a .-> a
  msfcata phi (Roll1 x)     = phi Inverse1 (msfcata phi)  x
  msfcata phi (Inverse1 z) = z
\end{lstlisting}

\textit{Note.}
The formulation of \lstinline{Rec1} and \lstinline{msfcata1} in
the previous work by Ahn and Sheard~\cite{AhnShe11} should be adjusted
as shown above. Although the previous formulation is type correct,
we realized that one cannot write useful examples over indexed datatypes
such as the type-preserving evaluator example in this paper. It was
an oversight due to the lack of testing their formulation
by examples over indexed mixed-variant datatypes.
\end{figure}

In Listing~\ref{lst:reccomb}, we illustrate the two recursion schemes,
\lstinline{mcata} and \lstinline{msfcata}, using Haskell.
We use a subset of Haskell, where we restrict the use of certain language
features and some of the definitions we introduce. We will explain
the details and motivation of these restrictions as we discuss
Listing~\ref{lst:reccomb}.

Each Mendler-style recursion scheme is described by a pair:
a type fixpoint (\eg, \lstinline{Mu0}, \lstinline{Rec0}) and
its constructors (\eg, \lstinline{In0}, \lstinline{Roll0}),
and the recursion scheme itself (\eg, \lstinline{mcata0}, \lstinline{msfcata0}).
A Mendler-style recursion scheme is characterized by
the abstract operations it supports. The types of
these abstract operations are evident in the type signature
of the recursion scheme. In Listing~\ref{lst:reccomb},
we emphasize this by factoring out the type of the first argument ($\varphi$)
as a type synonym prefixed by \lstinline{Phi}. Note the various synonyms
for each recursion scheme -- \lstinline{Phi0} has one abstract operation
and \lstinline{Phi0'} has two.

Mendler-style recursion schemes take two arguments.
The first is a function\footnote{By convention,
	we denote the function as $\varphi$. Which is why
	the type synonyms are prefixed by \lstinline{Phi}.}
that will be applied to concrete implementations of the abstract operators,
then uses these operations to describe the computation.
The second argument is a recursive value to compute over.
One programs by supplying specific instances of the first argument $\varphi$.

\subsection{Mendler-style iteration}
\label{sec:mendler:it}
Mendler-style iteration (\MIt{}) operates on recursive types constructed by
the fixpoint $\mu$. The fixpoint $\mu$ is indexed by a kind. We describe
$\mu$ at kind $*$ and $*\to*$ in Listing~\ref{lst:reccomb}.
We enforce two restrictions on the Haskell code in the Mendler style examples:
\begin{itemize}
\item Recursion is allowed only in the definition of the fixpoint at type-level,
and in the definition of the recursion scheme at term-level.
The type constructor \lstinline{Mu0} expects a non-recursive base structure
\lstinline{f :: * -> *} to construct a recursive type \lstinline{(Mu0 f :: *)}.
The type constructor \lstinline{Mu1} expects a non-recursive base structure
\lstinline{f :: (* -> *) -> (* -> *)} to construct a recursive type constructor
\lstinline{(Mu1 f :: * -> *)}, which expects one type index
(\lstinline{i :: *}). We do not use recursive datatype definitions (as
natively supported by Haskell) elsewhere. We do not use recursive function
definitions either, except to define Mendler-style recursion schemes.

\item Elimination of recursive values is only allowed via the recursion scheme.
One is allowed to freely introduce recursive values using \In{}-constructors,
but not allowed to freely eliminate (\ie, pattern match against \In{})
those recursive values. Note that \lstinline{mcata0} and \lstinline{mcata1}
are defined using pattern matching against \In{*} and \In{*\to*}.
Pattern matching against them elsewhere is prohibited.
\end{itemize}

The type synonyms \lstinline{Phi0} and \lstinline{Phi1} describe the types of
the first arguments of \lstinline{mcata0} and \lstinline{mcata1}.
These type synonyms indicate that Mendler-style iteration supports
one abstract operation: abstract recursive call \lstinline{(r -> a)}.
The type variable $r$ stands for an abstract recursive value, which could
be supplied to the abstract recursive call as an argument. Since $r$ is
universally quantified within \lstinline{Phi0} and \lstinline{Phi1},
functions of type \lstinline{Phi0 f a} and \lstinline{Phi1 f a} must be
parametric over $r$ (\ie, must not rely on examining any details of $r$-values).
In \lstinline{Phi0}, \lstinline{(r -> a)} is the type for
an abstract recursive call, which computes an answer of type \lstinline{a}
from the abstract recursive type \lstinline{r}.
This abstract recursive call is used to implement a function of type
\lstinline{f r -> a}, which computes an answer (\lstinline{a}) from
\lstinline{f}-structures filled with abstract recursive values (\lstinline{r}).
Similarly, \lstinline{(forall i.r i -> a i)} in \lstinline{Phi1} is the type
for an abstract recursive call, which is an index preserving function that
computes an indexed answer \lstinline{(a i)} from an indexed recursive value
\lstinline{(r i)}. In the Haskell definitions of \lstinline{mcata0} and
\lstinline{mcata1}, these abstract operations are made concrete by
a native recursive call. Note that the first arguments to
\lstinline{phi} in the definitions of \lstinline{mcata0} and \lstinline{mcata1}
are \lstinline{(mcata0 phi)} and \lstinline{(mcata1 phi)}.

Uses of Mendler-style recursion schemes are best explained by examples.
Listing~\ref{lst:Len} is a well-known example of a list length function
defined in terms of \lstinline{mcata0}. The recursive type for lists
\lstinline{(List  p)} is defined as a fixpoint of \lstinline{(L p)},
where \lstinline{L} is the base structure for lists. The data constructors
of \lstinline{List}, \lstinline{nil} and \lstinline{cons}, are defined
in terms of \lstinline{In0} and the data constructors of \lstinline{L}.
We define \lstinline{length} by applying \lstinline{mcata0} to
the \lstinline{phi} function. The function \lstinline{phi} is defined
by two equations, one for the \lstinline{N}-case and the other for
the \lstinline{C}-case. When the list is empty (\lstinline{N}-case),
the \lstinline{phi} function simply returns 0. When the list has an
element (\lstinline{C}-case), we first compute the length of the tail
(\ie, the list excluding the head, that is, the first element) by
applying the abstract recursive call \lstinline{(len :: r -> Int)}\footnote{
	Here, the answer type is \lstinline{Int}. }
to the (abstract) tail \lstinline{(xs :: r)},\footnote{
	Note that \lstinline{C x xs :: L p r} since \lstinline{xs :: r}.}
and, then, we add 1 to the length of the tail \lstinline{(len xs)}.

\begin{figure}
\lstinputlisting[
	caption={List length example using \lstinline{mcata0} \label{lst:Len}},
	firstline=3]{Len.hs}
\vspace*{-3ex}
\end{figure}


\subsection{Mendler-style iteration with syntactic inverses}
\label{sec:mendler:sf}
Mendler-style iteration with syntactic inverses (\msfit{}) operates on
recursive types constructed by the fixpoint $\mu'$. The fixpoint $\mu'$
is a non-standard fixpoint additionally parametrized by the answer type ($a$)
and has two constructors $\In{}'$ and \textit{Inverse}. $\In{}'$-constructors
are analogous to \In{}-constructors of $\mu$. \textit{Inverse}-constructors
hold answers to be computed by \msfit{}. For example,% \footnote{
	% In fact, this example is ill typed, because \msfit{} expects
	% its second argument type to be parametric over
	% (\ie, does not rely on specifics of) the answer type.
	% This example is just to illustrate the intuitive idea.}
the result of computing \lstinline{ msfcata  phi (Inverse0 5) }
is \lstinline{5} regardless of \lstinline{phi}.
The stylistic restrictions on the Haskell code involving \msfit{} are:
\begin{itemize}
\item Recursion is only allowed by the fixpoint at type-level ($\mu'$)
and by the recursion scheme (\msfit{}) at term-level.
We do not rely on recursive datatype definitions and function definitions
defined by the general recursion natively supported in Haskell.
\item Elimination of recursive values is allowed via the recursion scheme.
One is allowed to freely construct recursive values using $\In{}'$-constructors,
but not allowed to freely eliminate (\ie, pattern match against $\In{}'$) them.
Pattern matching against \textit{Inverse} is also forbidden.
\end{itemize}
These restrictions are similar to the stylistic restrictions involving \MIt{}.


The abstract operations supported by \msfit{} are evident
in the first argument type -- \lstinline{Phi0'} and \lstinline{Phi1'}
are the type synonyms for the first argument types of \lstinline{msfcata0}
and \lstinline{msfcata1}. Note that the abstract recursive type $r$ is also
additionally parametrized by the answer type $a$ in the type signatures
of \lstinline{msfcata0} and \lstinline{msfcata1}, since $\mu'$ is additionally
parametrized by $a$. In addition to the abstract recursive call, \msfit{}
also supports the abstract inverse operation. Note that the types for
abstract inverse (\lstinline{(a -> r a)} and \lstinline{(a i -> r a i)})
are indeed the types for inverse functions of abstract recursive call
(\lstinline{(r a -> a)} and \lstinline{(r a i -> a i)}). Instead of using
actual inverse functions to compute inverse images from answer values
during computation, one can hold intermediate answer values, whose inverse
images are irrelevant, inside \textit{Inverse}-constructors during
the computation using \msfit{}.

The type signature of \msfit{} expects the second argument to be
parametric over the answer type. Note the second argument types
\lstinline{(forall a. Rec0 f a)} and \lstinline{(forall a. Rec1 f a i)}
in the type signatures of \lstinline{msfcata0} and \lstinline{msfcata1}. 
Using \textit{Inverse} to construct recursive values elsewhere is, in a way,
prohibited due to the second argument type of \msfit{}. Using \textit{Inverse}
to construct concrete recursive values makes the answer type specific.
For example, \lstinline{(Inverse0 5) :: Rec0 f Int}, whose answer type
made specific to \lstinline{Int}, cannot be passed to \lstinline{msfcata0}
its second argument. The constructor \textit{Inverse} is only intended to
define \msfit{} and its first argument (\lstinline{phi}). One can indirectly
access \textit{Inverse} via the abstract inverse operation supported by
\msfit{}. Note, in the Haskell definitions of \lstinline{msfcata0} and
\lstinline{msfcata1}, the second arguments to \lstinline{phi} are
\lstinline{Inverse0} and \lstinline{Inverse1}. That is, the abstract inverse
operation is implemented by the \textit{Inverse}-constructor.

\begin{figure}
\lstinputlisting[
	caption={Formatting an untyped HOAS expression into a \lstinline{String}
		\label{lst:HOASshow} (adopted from \cite{AhnShe11}) },
	firstline=5]{HOASshow.hs}
\vspace*{-4ex}
\end{figure}

The HOAS formatting is a ``hello world'' example repeatedly formulated
in studies on recursion schemes over HOAS; \eg, \cite{FegShe96,bgb,BahHvi12}
to mention a few in the conventional style. This example is interesting because it
is a simplification of a recurring pattern (or functional pearl \cite{AxeCla13})
of conversion from higher-order syntax to first-order syntax, which is often
found in implementations of embedded domain specific languages.
Listing~\ref{lst:HOASshow} illustrates a Mendler-style formulation
(\lstinline{showExp}) of this example using \msfit{}.


The key characteristic of \lstinline{showExp} is apparent in the user-defined
combining function \lstinline{phi}. From the type of \lstinline{phi}, we know
that the result of iteration over a HOAS term \lstinline{e} is a function;
more specifically,
\lstinline{msfcata0  phi e :: [String ] -> String}. An infinite list of
fresh variable names (\lstinline{vars})\footnote{
	To be strictly complacent to the conventions of the Mendler style,
	we would have to formulate a co-recursive datatype
	that generates infinite list of variable names.
	Here, we simply use Haskell's lazy lists because our focus here is
	not co-recursion but introducing an example using \msfit{}.}
is supplied as an argument to \,\lstinline{msfcata0  phi e}\, to obtain
a string that represents \lstinline{e}.

Definition of \lstinline{phi} consists of two equations.
The first equation for \lstinline{App} is a typical structural recursion
over positive occurrences of recursive subcomponents. The second equation
for \lstinline{Lam} exploits the abstract inverse
(\lstinline{inv :: ([String ] -> String) -> r ([String ] -> String)})
provided by \msfit{} to handle the negative recursive occurrence.
When formatting a \lstinline{Lam}-expression, one should supply a fresh variable
to represent the bounded variable (which is the negative recursive occurrence)
introduced by \lstinline{Lam}. Here, we consume one fresh name from the supplied
list of fresh names by pattern matching \lstinline{(v:vs)}, and take an inverse
of a constant function that will return the name \lstinline{(inv(const  v))},
which has an appropriate type to pass into the function \lstinline{z} contained
in constructor \lstinline{Lam}. Since the result of this application
\lstinline{z(inv(const  v))} corresponds to a positive recursive occurrence,
we simply apply the abstract recursive call \lstinline{show'}.


\section{Type-preserving evaluation of the simply-typed HOAS}
\label{sec:HOASeval}
We can write an evaluator for a simply-typed HOAS in a simple manner
using \msfit{*\to*}, as illustrated in Listing\;\ref{lst:HOASeval}.
We first define the simply-typed HOAS as a recursive indexed datatype
\lstinline{Exp  :: * -> *}. We take the fixpoint using \lstinline{Rec1}
(the fixpoint with a syntactic inverse). This fixpoint is taken over
a non recursive base structure \lstinline{ExpF  :: (* -> *) -> (* -> *)}.
Note that \lstinline{ExpF} is an indexed type. So expressions will be indexed
by their type. Using \lstinline{Rec1} the fixpoint of any structure is also
parametrized by the type of the answer. The use of the \lstinline{msfcata1}
requires that \lstinline{Exp} should be parametric in this answer type
(by defining \lstinline{type Exp  t =  forall a. Exp' a}). 
%% just as we did
%% in the untyped HOAS formatting example in Listing\;\ref{lst:HOASshow}.

The definition of \lstinline{eval} specifies how to evaluate
an HOAS expression to a host-language value (\ie, Haskell) wrapped by
the identity type (\lstinline{Id}\,). In the description below, we ignore
the wrapping (\lstinline{MkId}\,) and unwrapping (\lstinline{unId}\,) of
\lstinline{Id} by completely dropping them from the description.
See the Listing\;\ref{lst:HOASeval} (where they are not omitted)
if you care about these details. We discuss the evaluation for each of
the constructors of \lstinline{Exp}:
\begin{itemize}
	\item Evaluating an HOAS abstraction (\lstinline{Lam f}\,) lifts
		an object-language function \lstinline{(f)} over \lstinline{Exp}
		into a host-language function over values:
		\lstinline{(\v -> ev (f(inv v)))}.
		In the body of this host-language lambda abstraction,
		the inverse of the (host-language) argument value \lstinline{v}
		is passed to the object-language function \lstinline{f}.
		The resulting HOAS expression \lstinline{(f(inv v))} is
		evaluated by the recursive caller (\lstinline{ev}) to
		obtain a host-language value.

	\item Evaluating an HOAS application \lstinline{(App f x)} lifts
		the function \lstinline{f} and argument \lstinline{x} to
		host-language values \lstinline{(ev f}) and \lstinline{(ev x)},
		and uses host-language application to compute
		the resulting value. Note that the host-language application
		\lstinline{((ev f) (ev x))} is type correct since
		\lstinline{ev f :: a -> b}\, and \lstinline{ev x :: a},
		thus the resulting value has type \lstinline{b}.
\end{itemize}
We know that \lstinline{eval} indeed terminates since \lstinline{Rec1} and
\lstinline{msfcata1} can be embedded into System~\Fw\ in manner similar to
the embedding of \lstinline{Rec0} and \lstinline{msfcata0} into System~\Fw.

\begin{figure}
\lstinputlisting[
	caption={Simply-typed HOAS evaluation using \lstinline{msfcata1}
		\label{lst:HOASeval}},
        firstline=4]{HOASeval.hs}
\vspace*{-3ex}
\end{figure}

Listing\;\ref{lst:HOASeval} highlights two advantages of the Mendler style over
the conventional style in one example. This example shows that the Mendler-style
iteration with syntactic inverses is useful for both \textit{negative} and
\textit{indexed} datatypes. \lstinline{Exp} in Listing\;\ref{lst:HOASeval} has
both negative recursive occurrences and type indices.

The \lstinline{showExp} example in Listing\;\ref{lst:HOASshow},
which we discussed in the previous section, has appeared in the work
of Fegaras and Sheard \cite{FegShe96}, written in the conventional style.
So, the \lstinline{showExp} example, only shows that the Mendler style is
as expressive as the conventional style (although it is
perhaps syntactically more pleasant than the conventional style).
Although it is possible to formulate such a recursion scheme over
indexed datatypes in the conventional style (\eg, simply-typed HOAS evaluation
example of Bahr and Hvited \cite{BahHvi12}), it is not quite elegant as
in the Mendler style because the conventional style is based on ad-hoc polymorphism.
In contrast, \lstinline{msfcata} is uniformly defined over indexed datatypes of
arbitrary kinds. Both \lstinline{msfcata1}, used in the \lstinline{eval},
and \lstinline{msfcata0}, used in the \lstinline{showExp}, have exactly
the same syntactic definition, differing only in their type signatures,
as illustrated in Listing\;\ref{lst:reccomb}.

\section{$\pmb{\mu'}$-fixpoint is a subtype of $\pmb{\mu}$-fixpoint}\label{sec:murec}
We discussed the usefulness of \msfit{} by the illustrating examples on HOAS.
If one is to design a language based on Mendler-style recursion schemes,
one would want to support as many useful recursion schemes available,
including \MIt{} and \msfit{}. One issue in such design is that we have
two different fixpoints $\mu$ and $\mu'$. The standard fixpoint $\mu$
does not come with syntactic inverses while $\mu'$ comes with
its syntactic inverse. It would be a bad design choice to provide 
two unrelated fixpoints and let users deal with them manually.
We would like to apply as many recursion schemes to one recursive value
without manual conversion.

\begin{figure}
\lstinputlisting[
	caption={Coercion from $\mu'$-values to $\mu$-values
		using \lstinline{msfcata0} \label{lst:rec2mu} },
	firstline=4]{Exp2Expr.hs}
\begin{lstlisting}[
	caption={An incomplete attempt to convert from $\mu$-values
		to $\mu'$-values \label{lst:mu2rec} } ]
msfcata'  :: Phi0' f a -> Rec0 f a -> a
msfcata' phi (Roll0 x)     = phi Inverse0 (msfcata' phi)  x
msfcata' phi (Inverse0 z) = z

exp'2expr :: Exp' Expr -> Expr  -- %\textcomment{i.e, %} Rec0 ExpF (Mu0 ExpF) -> Mu0 ExpF
exp'2expr = msfcata' phi where
  phi inv p2r (Lam f)    = In0(Lam((\x -> p2r (f (inv x)))))
  phi inv p2r (App e1 e2)  = In0(App (p2r e1) (p2r e2))

expr2exp' :: Expr -> Exp' Expr  -- %\textcomment{i.e., %} Mu0 ExpF -> Rec0 ExpF (Mu0 ExpF)
expr2exp' (In0(Lam f))   = Roll0 (Lam (\x->expr2exp' (f (exp'2expr x))))
expr2exp' (In0(App e1 e2)) = Roll0 (App (expr2exp' e1) (expr2exp' e2))
\end{lstlisting}
\vspace*{-3ex}
\end{figure}

We discovered a coercion from $\mu'$-values to $\mu$-values,
as illustrated in Listing\;\ref{lst:rec2mu}. In Listing\;\ref{lst:rec2mu},
we define a mapping from \lstinline{Exp}
(\ie, \lstinline{forall a.Rec0 ExpF  a})
to \lstinline{Expr} (\ie, \lstinline{Mu0 ExpF})
using \lstinline{msfcata0}, where
\lstinline{ExpF} is a base structure for the untyped HOAS.
Since we have two fixpoints, \lstinline{Rec0} and \lstinline{Mu0},
we can define two recursive datatypes from the base structure \lstinline{ExpF}.
One is \lstinline{Exp} defined as \lstinline{(forall a.Rec0 ExpF  a)} and
the other is \lstinline{Expr} defined as \lstinline{Mu0 ExpF}.
The function \lstinline{exp2expr  :: Exp -> Expr}\, implements the mapping
from \lstinline{Rec0}-based HOAS expressions to \lstinline{Mu0}-based
HOAS expressions. Note, \lstinline{exp2expr}\, is defined using
\lstinline{msfcata0}.  Since there exists an embedding of
\lstinline{Mu0} and \lstinline{msfcata0} into System~\Fw\ \cite{AhnShe11},
\lstinline{exp2expr}\, is admissible in System~\Fw. However, it is unlikely
that we can embed a coercion function for an arbitrary base structure $f$,
\lstinline{mu2rec :: (forall a.Rec0 f a) -> Mu0 f}, in System~\Fw\footnote{
	The discussions in \S\ref{sec:theory} on the embedding of \msfit{}
	suggests why the \lstinline{mu2rec} is unlikely to be embedded
	in System~\Fw, but its specific instances, such as
	\lstinline{exp2expr}, can be embedded in System~\Fw.}.

The converse coercion from $\mu$-values to $\mu'$-values
is not likely to exist in general, but the conversion might
be possible when the answer type of the $\mu'$-values (\eg, \lstinline{a}
in \lstinline{Rec0 ExpF  a}) has been specialized to the final answer value.
For instance, we attempted to convert from
\lstinline{Exp' Expr} to \lstinline{Expr}, rather than from \lstinline{Exp}
(\ie, \lstinline{forall a.Exp' a}) to \lstinline{Expr}.\footnote{
	Also note that \lstinline{a} in
	(\lstinline{Rec0}~\lstinline{ExpF}~\lstinline{a})
	in the type signature of $\msfit{}'$
	is not quantified, {c.f.} (\lstinline{(forall a.Rec0 f a)}
	in the type signature of $\msfit{*}$.}
We illustrate this idea in Listing~\ref{lst:mu2rec}, which is still
an incomplete attempt since there is no termination guarantee for
\lstinline{expr2exp'}. Note that \lstinline{expr2exp'} is not defined
using a Mendler-style recursion scheme but using general recursion.

The coercion from \lstinline{(forall a.Rec0 ExpF  a)}
to \lstinline{(Mu0 ExpF)} exists.
We conjecture that
it should be possible to derive a coercion function
from $\mu'$-values to $\mu$-values when given a specific instance of
the base structure.
Therefore, when designing a language based on Mendler-style
recursion schemes, we may support coercion from $\mu'$-values to $\mu$-values.


We believe that \lstinline{msfcata0} can
express more functions than \lstinline{mcata0} (e.g., \lstinline{showExp}
in Listing\;\ref{lst:HOASshow}). Then, it may be the case that the set of
values of \lstinline{(forall a.Rec0 f a)} is in fact more restrictive than
the set of values of \lstinline{(Mu0 f)}. The additional expressiveness of
\lstinline{msfcata0} may be a compensation for the restrictions on
the value of \lstinline{(forall a.Rec0 f a)}. In summary, 
\lstinline{(forall a.Rec0 f a)} is a subset of \lstinline{(Mu0 f)}.
We believe that this generalizes to arbitrary kinds other than $*$.


\section{Embedding \msfit{} into System~\Fw{}}\label{sec:theory}

We first review the embedding of Mendler-style iteration (\MIt{*}),
before discussing the embedding of Mendler-style iteration with
syntactic inverses (\msfit{*}). The embedding of Mendler-style iteration
consists of a polymorphic encoding of the fixpoint operator ($\mu_{*}$)
and term encodings (as functions) of its constructor ($\In{*}$)
and eliminator (\MIt{*}). We also show that one can derive
the equational properties of \MIt{*}, which correspond to
its Haskell definition discussed earlier.

Next, we discuss the embedding of \msfit{*}
into System~\Fw. The embedding of Mendler-style iteration with
syntactic inverses should consist of a polymorphic encoding of
the inverse-augmented fixpoint operator ($\mu_{*}'$) and term encodings of
its two constructors ($\textit{Inverse}_{*}$ and $\In{*}'$) and the eliminator
(\msfit{*}). The embedding is not as simple as the embedding of $\mu_{*}$
and $\MIt{*}$ because we have not found an \Fw-term that embeds $\In{*}'$.
However, we can embed each recursive type (\eg, \lstinline{Exp'}), when given
a concrete base structure (\eg, \lstinline{ExpF}), and deduce general rules
of how to embed inverse-augmented recursive types. We also show that
we can derive the expected equational properties for a specific example
(assuming that the section-retraction pair of the identity type is
equivalent to an identity function); the example we use is the untyped HOAS
(\lstinline{Exp'}) discussed in earlier sections.

Our discussion in this section is focused at kind $*$, but the embeddings for
Mendler-style recursion schemes at higher-kinds (\eg, \MIt{*\to*} and
\msfit{*\to*}) would be similar to the embeddings of them at kind $*$.
In fact, the term definitions for data constructors and eliminators
(\ie, recursion schemes) are always exactly the same regardless of their kinds.
Only their types become richer as we move to higher kinds, having more indices
applied to type constructors.

\subsection{The embedding of \MIt{*} and its equational property}
\label{sec:theory:mit}
Mendler-style iteration (\MIt{*}) can be embedded into
System~\Fw\ as follows \cite{AbeMatUus05,AhnShe11}:\vspace*{-1ex}
\begin{align*}
&\mu_{*} ~=~ \lambda F^{* \to *}.\forall X^{*}.
		(\forall R^{*}.(R \to X) \to F R \to X) \to X
	\\
&\MIt{*} ~:~ \forall A^{*}.
	(\forall R^{* \to *}.(R \to A) \to F R \to A) \to \mu_{*} F \to A
\\[-.5ex]
&\MIt{*}~\varphi~r ~=~ r~\varphi
	\\
&\In{*} ~:~\forall F^{* \to *}.F(\mu_{*} F) \to \mu_{*} F \\[-.5ex]
&\In{*}~x~\varphi ~=~ \varphi~(\MIt{*}~\varphi)~x\\[-5ex]
\end{align*}
From the above embedding, one can derive the equational property of \MIt{*}
apparent in the Haskell definition (Listing~\ref{lst:reccomb}) as follows:
$\MIt{*}~\varphi~(\In{*}~x) = \In{*}~x~\varphi = \varphi~(\MIt{*}~\varphi)~x$.

\subsection{Embedding \msfit{*}}\label{sec:theory:embed}
The aim is to embed Mendler-style iteration with static inverses (\msfit{*})
into System \Fw\ along the following lines.\footnote{
	A Haskell transcription of this embedding appears
	in the previous work of Ahn and Sheard \cite{AhnShe11}.}
The embeddings for $\mu_{*}'$ and \msfit{*} can given as follows:\vspace*{-1ex}
\begin{align*}
&\mu_{*}' ~=~ \lambda F^{* \to *}.\lambda A^{*}.
		K A + ((K A \to A) \to F(K A) \to A) \to A
	\\
&\msfit{*} ~:~ \forall A^{*}.
	(\forall R^{*\to*}.(A \to R A) \to (R A \to A) \to F(R A) \to A) \to
	(\forall A^{*}.\mu_{*}' F A) \to A \\[-.5ex]
&\msfit{*}~\varphi~r ~=~ r~\eta^{-1}~(\underbrace{\lambda f.f(\varphi~\eta)}_g)
	\\[-6ex]
\end{align*}
where $K = \lambda A^{*}.A$ is an identity type constructor, therefore,
both $\eta: A \to KA$ and $\eta^{-1}: KA \to A$ are identity functions.
We could have just erased $K$ in the embedding of $\mu_{*}'$ above,
but having $K$ makes it syntactically more evident of the correspondence
between this \Fw-embedding and the Haskell transcription
in Listing~\ref{lst:HOASeval}.\footnote{
	The purpose of identity datatype \lstinline{Id}
	in Listing~\ref{lst:HOASeval} is to avoid higher-order unification
	during type inference so that GHC can type check.}
It is also easier to notice that $KA$ matches with $RA$ through
polymorphic instantiation while type checking the definition of \msfit{*}.
In the embedding of $\msfit{*}$, note that $r : \mu_{*}'\,F\,A$ and that
$\mu_{*}'$ is defined using a sum type ($+$), whose polymorphic embedding is
$A + B = \forall X^{*}.(A \to X) \to (B \to X) \to X$ and its two constructors
$\inL : \forall A^{*}.\forall B^{*}.A \to A + B$ (left injection) and
$\inR : \forall A^{*}.\forall B^{*}.B \to A + B$ (right injection) are
defined as $\inL = \lambda a.\lambda f_1.\lambda f_2.f_1\;a$ and
$\inR = \lambda b.\lambda f_1.\lambda f_2.f_2\;b$. The value $r$ selects
$\eta^{-1} : KA \to A$ to handle $\textit{Inverse}_{*}$-values and selects
$g$ to handle $\In{*}'$-values.


Next, we need to embed the two data constructors of $\mu_{*}$,
$\textit{Inverse}_{*}$ and $\In{*}'$.

We were able to define a universal embedding of $\textit{Inverse}_{*}$
that works for arbitrary $F$:\vspace*{-1ex}
\begin{align*}
&\textit{Inverse}_{*} ~:~ \forall F^{*\to*}.\forall A^{*}.A \to \mu_{*}' F A \\
&\textit{Inverse}_{*}~a ~=~ \inL (\eta~a)
\end{align*}
From the embedding of $\textit{Inverse}_{*}$, we can derive the equational
property of \msfit{*} over $\textit{Inverse}_{*}$-values, which is apparent
in the Haskell definition of \msfit{*} in Listing\;\ref{lst:reccomb},
as below:\vspace*{-1ex}
\[
\msfit{*}~\varphi~(\textit{Inverse}_{*} a) =
(\textit{Inverse}_{*}~a)~\eta^{-1}~g
= \inL~(\eta~a)~\eta^{-1}~g = \eta^{-1}(\eta~a) = a
\]

However, we have not been able to define a universal embedding of $\In{*}'$
in System \Fw. What we know is that the embedding of $\In{*}'$ must be
in the form of a right injection ($\inR$):\vspace*{-1ex}\footnote{%
	It was also the case in the previous work of Ahn and Sheard
	\cite{AhnShe11}, but was not clearly stated in the text.}
\begin{align*}
&\In{*}' ~:~ \forall F^{*\to*}.\forall A^{*}.F(\mu_{*}' F A) \to \mu_{*}' F A \\
&\In{*}'~x ~=~ \inR(\,\cdots\;\text{\color{blue}missing complete definition}\;\cdots)
	\\[-5ex]
\end{align*}
We believe that we can find an embedding of $\In{*}'$ for each $F$ when
the definition of $F$ is given concretely (see Appendix~\ref{sec:appendix}).
That is, we can embed constructor functions of a recursive type $\mu{*}' F$
for each specific $F$.\footnote{
	Similarly, all regular recursive types can be embedded into System \F,
	but not $\mu_{*}$ itself.}
For instance, we can embed the constructor functions of \lstinline{Exp'}
in Listing\;\ref{lst:HOASshow} and its two data constructors \textit{lam}
and \textit{app} into System~\Fw, as below:\vspace*{-1ex}\footnote{
	The use of $\In{*}'$ here is only a conceptual illustration
	because we have embedded $\In{*}'$ itself into System~\Fw.
	We also labeled some of the subterms ($v$, $w$, and $h$)
	for later use in the discussion.}%
\label{align:embed}%
\begin{align*}
\textit{lam}&~:~\forall A^{*}.
		(\textit{Exp}'~A \to \textit{Exp}'~A) \to \textit{Exp}'~A
\\[-3ex]
\textit{lam}&~f~=~\In{*}'(\textit{Lam}~f)
= \inR~(\underbrace{
	\lambda \varphi'.\varphi'~\eta^{-1}~
	(\overbrace{\textit{Lam}(\lambda y.\textit{lift}~\varphi'~(f(\inL~y))) }^v)\;
	}_w)
\\[-1.5ex]
\textit{app}&~:~\forall A^{*}.
		\textit{Exp}'~A \to \textit{Exp}'~A \to \textit{Exp}'~A \\
\textit{app}&~r_1~r_2 = \In{*}'(\textit{App}~r_1~r_2)
= \inR~(\underbrace{
		\lambda \varphi'.\varphi'~\eta^{-1}~
		(\textit{App}~(\lift~\varphi'~r_1)~(\lift~\varphi'~r_2))}_h)
\end{align*}~\vspace*{-4.5ex}\\
where \textit{lift} is defined as follows:\vspace*{-1ex}
\begin{align*}
& \lift~:~(\forall A^{*}.(K A\to A)\to F(K A)\to A)\to \mu_{*}' F A\to K A\\
& \lift~\varphi'~r ~=~ r~\textit{id}~(\lambda z.\eta(z~\varphi'))
\end{align*}

Recall that $\mu_{*}'$ is a sum type. The \textit{lift} function converts
$(\mu_{*}'FA)$-values to $(KA)$-values when given a function
$\varphi' : \forall A^{*}.(KA \to A) \to F(K A) \to A$.
Observe that the type of $\varphi'$ matches with the partial application of
$\varphi$, the first argument of \textit{msfit}, applied to $\eta$. Since
$\varphi : \forall R^{*}.(A \to R A) \to (R A \to A) \to F(R A) \to A$
and $\eta : A \to K A$, we first instantiate $R$ with $K$ in the type of
$\varphi$, that is, $(A \to K A) \to (K A \to A) \to F(K A) \to A$.
Then, $(\varphi\eta) : (K A \to A) \to F(K A) \to A$, which matches
the type of $\varphi'$, the first argument of \textit{lift}.

We use \textit{lift} for the recursive values that are covariant,
in order to convert from $F(\mu_{*}'FA)$-structures, or $F(RA)$-structures,
to $F(KA)$-structures -- recall the type of the $\varphi'$.
We lift recursive values $r_1$ and $r_2$, which are both covariant,
in the embedding of \textit{app}. We also lift the value resulting from $f$,
whose return type is $F(\mu_{*}'FA)$, in the embedding of \textit{lam},
since the right-hand side of the function type is covariant.

For recursive values needed in contravariant positions, we simply left inject
answer values. For example, $y$ in the embedding of \textit{lam} has type $KA$
since we expect the argument to $Lam$ be of type $KA \to KA$
because we expect $v : F(KA)$, which is the second argument to be
applied to $\varphi'$. To convert from $(KA)$ to $\mu_{*}'FA$,
we only need to left inject, that is, $(\inL~y)$,
which can be applied to $f : \mu_{*}'FA \to \mu_{*}'FA$.

We believe that it is possible to give an embedding for any recursive type
in this way, that is, by lifting $(\textit{lift}~\varphi)$ the recursive values
in covariant positions and by left injecting ($\inL$) the answer values
when recursive values are needed in contravariant positions. A type-directed
algorithm for deriving the embeddings of the constructor functions of
$\mu'_{*} F$ for each given $F : * \to *$ is described
in Appendix~\ref{sec:appendix}). It would be an interesting theoretical quest
to search for a calculus that can directly embed the constructor
$\In{*}' : \forall F^{*\to*}.\forall A^{*}.F(\mu'_{*} F A) \to \mu'_{*} F A$.

In the remainder of this section, we discuss the equational properties
of \msfit{*} over $\In{*}'$-values of the type \lstinline{Exp}. That is,
when \msfit{*} is applied to the values constructed either by \textit{app}
or by \textit{lam}.

\subsection{Equational properties of \msfit{*} over
	values constructed by \textit{lam}}
	\label{sec:theory:eqlam}
When applied to $(\textit{lam}~f)$, we expect \msfit{*} to
satisfy the following equation:\vspace*{-1ex}
\begin{equation}
\msfit{*}~\varphi~(\textit{lam}~f)
\stackrel{?}{=} \varphi~\eta~\eta^{-1}
	~(\textit{Lam}(\lambda y.\eta(\msfit~\varphi~(f(\inL~y)))))
\label{eqn:msfitLam}
\end{equation}
We use $\eta$ to convert answer values of type $A$, resulting from
$(\msfit~\varphi~(f(\inL~y)))$, to values of type $KA$, since we need
$(\textit{Lam}(\lambda y.\eta(\msfit~\varphi~(f(\inL~y)))))$
to be of type $F(KA)$. The type of $\varphi$ expects a value of type $F(RA)$
as its third argument, where $R$ is a polymorphic type variable, which
instantiates to $K$ in the right-hand side of Equation~(\ref{eqn:msfitLam}).
We use $\inL$ to convert $y : KA$, to a value of $\mu_{*}' F A$
in order to apply it to $f : \mu_{*}' F A \to \mu_{*}' F A$.

The left-hand side of Equation~(\ref{eqn:msfitLam}) can be expanded
using the definitions of \msfit{*}, $\inR$, $g$, and $w$,
as below:\vspace*{-1ex}
\begin{align*}
\msfit{*}~\varphi~(\textit{lam}~f)
&~=~ (\textit{lam}~f)~\eta^{-1}~g \\
&~=~ \inR~w~\eta^{-1}~g ~=~ g~w ~=~ w(\varphi\eta) \\
&~=~ \varphi~\eta~\eta^{-1}~
	(\textit{Lam}(\lambda y.\lift~(\varphi\eta)~(f(\inL~y)))) \\
&~=~ \varphi~\eta~\eta^{-1}~(\textit{Lam}(\lambda y.\psi))
\end{align*}
where $\psi = (f(\inL~y))~\textit{id}~(\lambda z.\eta(z(\varphi\eta)))$.

The resulting equation is similar in structure to the right-hand side of
Equation~(\ref{eqn:msfitLam}). Thus, justifying Equation~(\ref{eqn:msfitLam})
amounts to showing:\vspace*{-1ex}
\begin{equation}
\psi \stackrel{?}{=} \eta(\msfit~\varphi~(f(\inL~y))))
\label{eqn:msfitPsi}
\end{equation}
The right-hand side of Equation~(\ref{eqn:msfitPsi}) expands as follows:\vspace*{-1ex}
\[ \eta(\msfit~\varphi~(f(\inL~y)))) = \eta(\inL~\psi~\eta^{-1}~g)
	= \eta(\eta^{-1}~\psi) = \psi
\]
In the last step of $\eta(\eta^{-1}\psi)=\psi$, we relied on the fact that
$\eta$ and $\eta^{-1}$ are identity functions.

\subsection{Equational properties of \msfit{*} over
	values constructed by \textit{app}}
	\label{sec:theory:eqapp}
When applied to $(\textit{app}~r_1~r_2)$, we expect \msfit{*} to
recurse on each of $r_1$ and $r_2$, as follows:\vspace*{-1ex}
\begin{equation}
\msfit{*}~\varphi~(\textit{app}~r_1~r_2)
\stackrel{?}{=}
\varphi~\eta~\eta^{-1}~(\textit{App}~(\eta(\msfit{*}~\varphi~r_1))
					~(\eta(\msfit{*}~\varphi~r_2)))
\label{eqn:msfitApp}
\end{equation}
We need $\eta$ to convert answer values of type $A$ to values of type $KA$,
since we need $(\textit{App}~(\eta(\msfit{*}~\varphi~r_1))
					~(\eta(\msfit{*}~\varphi~r_2)))$
to have type $F(KA)$. The type of $\varphi$ expects a value of type $F(RA)$
as its third argument, where $R$ is a polymorphic type variable, which
instantiates to $K$ in the right-hand side of Equation~(\ref{eqn:msfitApp}).
By using the definitions of \msfit{*}, $\inR$, $g$, and  $h$,
the left-hand side of Equation~(\ref{eqn:msfitApp}) expands as follows:\vspace*{-1ex}
\begin{align*}
\msfit{*}~\varphi~(\textit{app}~x~y)
&~=~ (\textit{app}~r_1~r_2)~\eta^{-1}~g \\
&~=~ \inR~h~\eta^{-1}~g ~=~ g~h ~=~ h(\varphi~\eta) \\
&~=~ \varphi~\eta~\eta^{-1}
     ~(\textit{App}~(\lift~(\varphi\eta)~r_1)~(\lift~(\varphi\eta)~r_2))
\end{align*}
The resulting expression is similar in structure to the right-hand side of
Equation~(\ref{eqn:msfitApp}). Thus, justifying Equation~(\ref{eqn:msfitApp})
amounts to showing:\vspace*{-1ex}
\begin{equation}
\eta(\msfit{*}~\varphi~r) \stackrel{?}{=} \lift~(\varphi\eta)~r
\label{eqn:msfitLift}
\end{equation}
When $r = \inR~z$, Equation~(\ref{eqn:msfitLift}) is justified as follows:\vspace*{-1ex}
\begin{align*}
\eta(\msfit{*}~\varphi~(\inR~z))
&~=~ \eta(\inR~z~\eta^{-1}~g) ~=~ \eta(g~z) ~=~ \eta(z(\varphi\eta)) \\
&~=~ (\inR~z)~\textit{id}~(\lambda z.\eta(z.(\varphi\eta)))
 ~=~ \lift~(\varphi\eta)~(\inR~z)
\end{align*}
When $r = \inL~z$,
the right-hand side of Equation~(\ref{eqn:msfitLift}) expands as below:\vspace*{-1ex}
\[ \lift~\varphi~(\inL~z)
   ~=~ (\inL~z)~\textit{id}~(\lambda z.\eta(z.(\varphi\eta)))
   ~=~ \textit{id}~z = z
\]
and the left-hand side of Equation~(\ref{eqn:msfitLift}) expands as below
\[ \eta(\msfit{*}~\varphi~r) ~=~\eta(\inL~z~\eta^{-1}~g) = \eta(\eta^{-1}z) = z
\]
In the last step of $\eta(\eta^{-1}z)=z$, we relied on the fact that
$\eta$ and $\eta^{-1}$ are identity functions.


\section{Related work}\label{sec:relwork}
Here, we discuss several related work. In \S\ref{sec:relwork:mpr},
we introduce Mendler-style primitive recursion (\mpr{}) to lead up
the discussion of \mprsi{} (\S\ref{sec:ongoing:mprsi}). 
In \S\ref{sec:relwork:sized}, we summarize type-based termination
and sized-type approach (in relation to \mpr{}). Lastly,
in \S\ref{sec:relwork:PCDT}, we discuss a generic programming library
in Haskell that supports binders using parametric HOAS, which leads up
the discussion of \mphit{} (\S\ref{sec:ongoing:mphit}).
We also mention recent breakthrough regarding self-evaluation of System \Fw\ 
(\S\ref{ssec:relwork:selfinterp}).

\subsection{Mendler-style primitive recursion}\label{sec:relwork:mpr}
\begin{figure}
\lstinputlisting[
	caption={Examples using Mendler-style primitive recursion
		\mpr{} at kind $*$\,: a constant time \lstinline{pred} and
		a \lstinline{factorial} function. \label{lst:mprim}},
        firstline=12]{Fac.hs}
\vspace*{-3ex}
\end{figure}

Termination of the Mendler-style iteration (\MIt{}) can be proved by embedding
\MIt{} into System \Fw\ as discussed in \S\ref{sec:mendler:it}. The embedding
of \MIt{} in \S\ref{sec:mendler:it} is \emph{reduction preserving}:
the number of reduction steps using the embedding and using
the equational specification should differ no more than constant time factor.
A reduction preserving embedding of primitive recursion
into System \Fw\ cannot exist because it is known that
``induction is not derivable in second order dependent type theory''
\cite{Geuvers01} and that ``primitive recursion can be seen as
the computational interpretation of induction through
the Curry-Howard interpretation of propositions-as-types'' \cite{Hallnas92}.
Although it is possible to simulate primitive recursion in terms of iteration,
it may become computationally inefficient. For example, \lstinline{pred}\,
in Listing~\ref{lst:mprim} could be defined using \MIt{} but its
time complexity would be at least linear to the size of the input
rather than constant. A constant time \lstinline{pred} is definable due to
the abstract cast operation provided by \mpr{}. This operation casts
abstract recursive values of type $r$ into concrete recursive values
of type \lstinline{Mu0 f}\,; its type \lstinline{(r -> Mu0 f)} is apparent
from the type signature of \lstinline{mprim0}. This cast operation computes
in constant time because it is implemented as the identity function
(\lstinline{id}) in the definition of \lstinline{mprim0}. A representative
example of \mpr{} that actually uses recursion is the factorial function.
The multiplication function \lstinline{times} used in the definition of
\lstinline{factorial} can be defined in terms of \MIt{*} and
an addition function; in turn, the addition function can be defined
in terms of \MIt{*} as well. Mendler-style primitive recursion
generalizes to higher kinds in the same manner as \MIt{} and \msfit{}
(see Listing~\ref{lst:reccomb} in \S\ref{sec:mendler}).

Abel and Matthes \cite{AbeMat04} discovered a reduction preserving
embedding of the Mendler-style primitive recursion in System \Fixw,
which is a strongly normalizing calculus extending System \Fw\ with
polarized kinds and equi-recursive fixpoints. Polarized kinds extend
the kind arrow with polarities in the form of $p\,\kappa_1 \to \kappa_2$
where polarity $p$ is either $+$, $-$, or $0$; meaning that the argument
must be used in positive, negative, or any position, respectively.
For example, in a polarized system, the base structure \lstinline{N :: * -> *}
for natural numbers in Listing~\ref{lst:mprim} could be assigned $+*\to*$
because its argument $r$ is only used covariantly, and, base structure
\lstinline{ExpF  :: * -> *} in Listing~\ref{lst:HOASshow} for the untyped HOAS
(see \S\ref{sec:mendler:sf}) must be assigned kind $0* \to *$ because
its argument $r$ is used in both covariant and contravariant positions.
The equi-recursive fixpoint $\textsf{fix}_\kappa : +\kappa \to \kappa$
in System~\Fixw\ can be applied only to positive base structures.\footnote{
	Otherwise, equi-recursive types are able to express diverging
	computations when they are not restricted to positive polarity.}
Abel and Matthes encoded the iso-recursive fixpoint operator $\mu$
in terms of the equi-recursive fixpoint operator \textsf{fix}, by converting
base structures of arbitrary polarities into base structures of
positive polarities, in order to embed \mpr{} into System \Fixw.

\subsection{Type-based termination and sized types}\label{sec:relwork:sized}
\emph{Type-based termination} (coined by Barthe and others \cite{BartheFGPU04})
stands for approaches that integrate termination into type checking,
as opposed to syntactic approaches that reason about termination over
untyped term structures. The Mendler-style approach is, of course,
type-based. In fact, the idea of type-based termination was inspired by
Mendler \cite{Mendler87,Mendler91}. In the Mendler style, we know that
well-typed functions defined using Mendler-style recursion schemes always
terminate. This guarantee follows from the design of the recursion scheme,
where the use of higher-rank polymorphic types in the abstract operations
enforce the invariants necessary for termination.

Abel \cite{abel06phd,Abel12talkFICS} summarizes the advantages of
type-based termination as:
\emph{communication} (programmers think using types),
\emph{certification} (types are machine-checkable certificates),
\emph{a simple theoretical justification}
        (no additional complication for termination other than type checking),
\emph{orthogonality} (only small parts of the language are affected,
        \eg, principled recursion schemes instead of general recursion),
\emph{robustness} (type system extensions are less likely to
                        disrupt termination checking),
\emph{compositionality}
        (one needs only types, not the code, for checking the termination), and
\emph{higher-order functions and higher-kinded datatypes}
(works well even for higher-order functions and non-regular datatypes).
In his dissertation~\cite{abel06phd} (Section 4.4) on sized types,
Abel views the Mendler-style approach as enforcing size restrictions
using higher-rank polymorphism as follows:
\begin{itemize}
\item The abstract recursive type $r$ in the Mendler style corresponds to
        $\mu^\alpha F$ in his sized-type system (System \Fwhat),
        where the sized type
        for the value being passed in corresponds to $\mu^{\alpha+1} F$.
\item The concrete recursive type $\mu F$ in the Mendler style corresponds to
        $\mu^\infty F$ since there is no size restriction.
\item By subtyping, a type with a smaller size-index can be cast to
        the same type with a larger size-index.
\end{itemize}
The same intuition holds for the termination behaviors
of Mendler-style recursion schemes over positive datatypes.
For positive datatypes, Mendler-style recursion schemes terminate
because $r$-values are direct subcomponents of the value being eliminated.
They are always smaller than the value being passed in.
Types enforce that recursive calls are only well-typed,
when applied to smaller subcomponents.

Abel's System \Fwhat\ can express primitive recursion quite naturally
using subtyping. The casting operation $(r \to \mu F)$ in Mendler-style
primitive recursion corresponds to an implicit conversion by subtyping
from $\mu^\alpha F$ to $\mu^\infty F$ because $\alpha \leq \infty$.
System \Fwhat\ \cite{abel06phd} is closely related to
System \Fixw\ \cite{AbeMat04}. Both of these systems are base on
equi-recursive fixpoint types over positive base structures.
Both of these systems are able to embed (or simulate) Mendler-style
primitive recursion (which is based on iso-recursive types) via
the encoding \cite{Geu92} of arbitrary base structures into
positive base structures.

Abel's sized-type approach evidences good intuition concerning the reasons
that certain recursion schemes terminate over positive datatypes.
But, we have not gained a useful intuition of whether or not those
recursion schemes would terminate for negative datatypes, unless there is
an encoding that can translate negative datatypes into positive datatypes.
For primitive recursion, this is possible (as we mentioned above). However,
for our recursion scheme \MsfIt, which is especially useful over negative
datatypes, we do not know of an appropriate encoding that can map
the inverse-augmented fixpoints into positive fixpoints. So, it is not clear
whether the sized-type approach based on positive equi-recursive fixpoints
can provide a good explanation for the termination of \MsfIt.

In \S\ref{sec:ongoing:mprsi}, we will discuss
another Mendler-style recursion scheme (\mprsi{}), which is also useful over
negative datatypes and believed to have a termination property
(not yet proved) based on the size of the index in the datatype.


\subsection{Parametric compositional data types}
\label{sec:relwork:PCDT}
Bahr and Hvited developed a generic programming library in Haskell,
\emph{compositional data types} (CDT)~\cite{bahr11wgp}, which builds on
Wouter Swierstra's ideas of \emph{data types \`a la carte}~\cite{WSout08jfp}.
Recently, they extended CDT to handle binders by adopting Adam Chlipala's
idea of PHOAS~\cite{PHOAS}, naming thier new extension as
\emph{parametric compositional data types} (PCDT).
In Section~3 of their paper on PCDT~\cite{BahHvi12}, they give an enlightening
comparative summary on a series of studies on recursion schemes over
mixed-variant datatypes in the conventional setting ---
Meijer and Hutton \cite{MeiHut95}, Fegaras and Sheard \cite{FegShe96},
Washburn and Weirich \cite{bgb}, and their own.

PCDT is based on the conventional style, relying on ad-hoc polymorphism.
That is, they need to derive a class instance of an appropriate algebra
in order to define a desired recursion scheme for each datatype definition.
For example, a functor instance for iteration and a difunctor (or profunctor)
instance for iteration with inverses over regular datatypes.
Since conventional-style recursion schemes do not generalize naturally
to non-regular datatypes such as GADTs, they also need to derive different
class instances, that is, higher-order functor and difunctor instances for
non-regular datatypes. To alleviate this drawback of the conventional style,
they automate instance derivation by meta-programming using Template Haskell
for the PCDT library user.

On the contrary, the Mendler style, being based on higher-order
parametric polymorphism, enjoys uniform definitions of recursion schemes
across arbitrary kinds of datatypes, naturally generalizing from regular
to non-regular datatypes. In \S\ref{sec:ongoing:mphit}, we demonstrate
this elegance of the Mendler style by formulating a Mendler-style counterpart of
the conventional-style recursion scheme in PHOAS. Here, we summarize
the key idea how Bahr and Hvited \cite{BahHvi12} factored out
the fixpoint operator from recursive formulations of PHOAS,\footnote{
	An online posting of Edward Kmett \cite{PHOASforFree}
	also discusses PHOAS in a formulation very similar to PCDT.}
in order to lead up the discussion in \S\ref{sec:ongoing:mphit}.

In PCDT, they transfer the essence of PHOAS using two-level types
that are equipped with an extra parameter in base functors as well as
in the fixpoint operator. For example, their fixpoint operator and
the base functor for the untyped HOAS would be defined as:\footnote{
	Bahr and Hvited named their fixpoint operator \textit{Trm} in PCDT.
	Here, we call it $\hat\mu$ in order to use a notation similar to
	the other operators ($\mu$ and $\breve\mu$) in this paper.
	In addition, they compose base functors with multiple constructors
	such as \textit{ExpF} from several single constructor functors;
	hence, their library is named \emph{compositional}. Here, we focus
	the discussions on the \emph{parametric} flavor of their contribution.}
\begin{lstlisting}
data Mu_0 (f :: * -> * -> *) a = In_0 (f a (Mu_0 f a)) | Var_0 a
data ExpF r_ r = Lam (r_ -> r) | App r r
\end{lstlisting}
Their fixpoint operator \lstinline{Mu_0} takes a type constructor of kind
\lstinline{* -> * -> *} as an argument, unlike the previously discussed
fixpoint operators (\eg, \lstinline{Mu0} or \lstinline{Rec0}) that take
arguments of kind \lstinline{* -> *}. Note the use of two parameters
\lstinline{r_} and \lstinline{r} used in contravariant and covariant positions
respectively in the definition of \lstinline{ExpF}; the additional
parameter \lstinline{r_} is used in a contravariant recursive position
in the argument of the \lstinline{Lam} constructor.

Then, the recursive type for the untyped HOAS is defined as
the fixpoint of base \lstinline{ExpF}:
\begin{lstlisting}
type Exp' a = Mu_0 ExpF a   -- %\textcomment{pre-expressions that may contain $\textit{Var}_{*}$%}
type Exp = forall a. Exp' a   -- %\textcomment{\!$\textit{Var}_{*}$-free expressions enforced by parametricty%}
\end{lstlisting}
When \lstinline{ExpF}
is applied to \lstinline{Mu_0}, the parameter \lstinline{r_} matches with
the answer type \lstinline{a} and the parameter \lstinline{r} matches with
the recursive type \lstinline{(Mu_0 ExpF  a)}.
Their \lstinline{Var_0} constructor for \lstinline{Mu_0} serves
the same purpose (\ie, injecting an inverse of an answer value)
as our \lstinline{Inverse0} for \lstinline{Rec0}.

Finally, the constructor functions for the untyped HOAS are defined as follows:
\begin{lstlisting}
lam f = In_0 (Lam (f . Var_0)) -- :: (Mu_0 f a -> Mu_0 ExpF a) -> Mu_0 ExpF a
app e1 e2 = In_0 (App e1 e2)     -- :: Mu_0 ExpF a -> Mu_0 ExpF a -> Mu_0 ExpF a
\end{lstlisting}
Note the similarities between the types of the constructor functions above
and the types of the constructor functions in Listing~\ref{lst:HOASshow}.
A notable difference is where the inverse injection is used:
their \lstinline{Var_0} is used in the constructor function implementation
(\lstinline{lam}), while our \lstinline{Inverse0} is used in
the recursion scheme implementation (\msfit{*}).

\subsection{Self-interpreter of System \Fw}\label{ssec:relwork:selfinterp}
Recently, there has been a breakthrough in normalization barrier of
defining a self-interpreter within a strongly normalizing language.
Previously, it was believed that self-interpreters were definable 
only in Turing-complete languages. Brown and Palsberg \cite{SelfInterpFomega}
successfully defined a self-interpretation of the System \Fw\ within System \Fw.
Interestingly, they also used HOAS repesentation of terms 
and a subset of Haskell (which is believed to be a subset of \Fw) to
semi-formally prove their theories, similarly to the previos work of
\cite{AhnShe11}. In perspective of this recent breaktrough, the existence of
an \Fw-embedding for \lstinline{msfit}, which can express simply-typed HOAS
evaluation, is indeed probable.


\section{Ongoing work}
\label{sec:ongoing}
We have two threads of ongoing work regarding Mendler-style recursion schemes
over mixed-variant datatypes ---
Mendler-style primitive recursion with a sized index (\S\ref{sec:ongoing:mprsi})
and
Mendler-style parametric higher-order iteration (\S\ref{sec:ongoing:mphit}).

\subsection{Mendler-style primitive recursion with a sized index}
\label{sec:ongoing:mprsi}

In \S\ref{sec:mendler} and \S\ref{sec:HOASeval}, we discussed
Mendler-style iteration with a syntactic inverse, \lstinline{msfcata},
which is particularly useful for defining functions over
negative (or mixed-variant) datatypes. We demonstrated the usefulness of
\lstinline{msfcata} by defining functions over HOAS:
\begin{itemize}
\item the string formatting function \lstinline{showExp} for
	the untyped HOAS using \lstinline{msfcata0}
	(Figure\;\ref{lst:HOASshow}) and
\item the type-preserving evaluator \lstinline{eval} for
	the simply-typed HOAS using \lstinline{msfcata1}
	(Figure\;\ref{lst:HOASeval}).
\end{itemize}

In this subsection, we speculate about another Mendler-style recursion scheme,
\lstinline{mprsi}, motivated by an example similar to the \lstinline{eval}
function. The name \lstinline{mprsi} stands for
Mendler-style primitive recursion with a sized index.

We review the \lstinline{eval} example and then compare
it to our motivating example \lstinline{veval} for \lstinline{mprsi}.
Both \lstinline{eval} and \lstinline{veval} are illustrated
in Figure\;\ref{lst:HOASevalV}. Recall that this code is written in Haskell,
following the Mendler-style conventions.
The function \lstinline{eval :: Exp t -> Id t} is
a type preserving evaluator that evaluates an HOAS expression of type
\lstinline{t} to a (Haskell) value of type \lstinline{t}.
The \lstinline{eval} function always terminates because
\lstinline{msfcata1} always terminates. Recall that \lstinline{msfcata1}
and \lstinline{Rec1} can be embedded into System~\Fw.

\begin{figure}
\lstinputlisting[
	caption={A simply-typed HOAS evaluation via a user-defined value domain.
		\label{lst:HOASevalV}
		},
	firstline=4]{HOASevalV.hs}
\vspace*{-3ex}
\end{figure}

The motivating example \lstinline{veval :: Exp t -> Val t} is also 
a type-preserving evaluator. Unlike \lstinline{eval}, it evaluates to 
a user-defined value domain \lstinline{Val} of type \lstinline{t} (rather
than a Haskell value). The definition of \lstinline{veval} is similar to
\lstinline{eval}; both of them are defined using \lstinline{msfcata1}.
The first equation of \lstinline{phi} for evaluating
the \lstinline{Lam}-expression is essentially the same as
the corresponding equation in the definition of \lstinline{eval}.
The second equation of \lstinline{phi} for evaluating
the \lstinline{App}-expression is also similar in structure to
the corresponding equation in the definition of \lstinline{eval}.
However, the use of \lstinline{unVal} is problematic. In particular,
the definition of \lstinline{unVal} relies on pattern matching against
\lstinline{In1}. Recall that one cannot freely pattern match against
a recursive value in the Mendler style. Recursive values must be analyzed
(or eliminated) by using Mendler-style recursion schemes. It is not a problem
to use \lstinline{unId} in the definition of \lstinline{eval} because
\lstinline{Id} is non-recursive.

It is not likely that \lstinline{unVal} can be defined using any of
the existing Mendler-style recursion schemes. So, we designed
a new Mendler-style recursion scheme that can express \lstinline{unVal}.
The new recursion scheme \lstinline{mprsi} extends \lstinline{mprim} with
an additional uncast operation. Recall that \lstinline{mprim} has
two abstract operations, call and cast. So, \lstinline{mprsi} has
three abstract operations, call, cast, and uncast. In the following paragraphs,
we explain the design of \lstinline{mprsi} step-by-step.

Let us try to define \lstinline{unVal} using \lstinline{mprim1} and examine
where it falls short. \lstinline{mprim1} provides two abstract operations,
\lstinline{cast} and {call}, as it can be seen from the type signature below:
\begin{lstlisting}
mprim1 :: (forall r i. (forall i. r i -> Mu1 f i)  -- cast
              -> (forall i. r i -> a i)      -- call
              -> (f r i -> a i)        ) -> Mu1 f i -> a i
\end{lstlisting}
We attempt to define \lstinline{unVal} using \lstinline{mprim1} as follows:
\begin{lstlisting}
unVal :: Mu1 V (t1 -> t2) -> (Mu1 V t1 -> Mu1 V t2)
unVal = mprim1 phi where
  phi cast call (VFun f) = ...
\end{lstlisting}
Inside the \lstinline{phi} function, we have a function
\lstinline{f :: (r t1 -> r t2)} over abstract recursive values.
We need to cast \lstinline{f} into a function over concrete recursive values
\lstinline{(Mu1 V t1 -> Mu1 V t2)}.
We should not need to use \lstinline{call}, since we do not expect
to use any recursion to define \lstinline{unVal}.
So, the only available operation is
\lstinline{cast :: (forall i.r i -> Mu1 f i)}.
Composing \lstinline{cast} with \lstinline{f}, we can get
\lstinline{(cast . f) :: (r t1 -> Mu1 V t2)}, whose codomain
\lstinline{(Mu1 V t2)} is exactly what we want. But, the domain
is still abstract \lstinline{(r t1)} rather than being concrete
\lstinline{(Mu1 V t1)}. We are stuck.

What additional abstract operation would help us complete
the definition of \lstinline{unVal}? We need an abstract operation
to cast from \lstinline{(r t1)} to \lstinline{(Mu1 V t1)}
in a contravariant position. If we had an inverse of cast,
\lstinline{uncast :: (forall i.Mu1 f i -> r i)}, we can
complete the definition of \lstinline{unVal} by composing
\lstinline{uncast}, \lstinline{f}, and \lstinline{cast}.
That is, \lstinline{uncast . f . cast :: (Mu1 V t1 -> Mu1 V t2)}.
Thus, we can formulate \lstinline{mprsi1} with a naive type signature
as follows:
\begin{lstlisting}
mprsi1 :: (forall r i. (forall i. r i -> Mu1 f i)  -- cast
               -> (forall i. Mu1 f i -> r i)  -- uncast
               -> (forall i. r i -> a i)      -- call
               -> (f r i -> a i)        ) -> Mu1 f i -> a i

mprsi1 phi (In1 x) = phi id id (mprsi1 phi) x
\end{lstlisting}
Although the type signature above is type-correct, it is too powerful.
The Mendler-style uses types to forbid non terminating computations
as ill-typed. Having both \lstinline{cast} and \lstinline{uncast} supports
the same ability as freely pattern matching over recursive values,
which can lead to non-termination. To recover the guarantee of termination,
we need to restrict the use of either \lstinline{cast} or \lstinline{uncast},
or both.

Let us see how this non-termination might occur. If we allowed
\lstinline{mprsi1} with the naive type signature above, we could write
an evaluator (similar to \lstinline{veval} but for an untyped HOAS),
which does not always terminate. This evaluator would diverge for terms
with self application.
Here, We walk through the process of defining an untyped HOAS.
The base structures of the untyped HOAS and its value domain
can be defined as follows:
\begin{lstlisting}
data ExpF_u r t = Lam_u (r t -> r t) | App_u (r t) (r t)
data V_u r t = VFun_u (r t -> r t)
\end{lstlisting}
Fixpoints of the structures above represent the untyped HOAS and
its value domain. Here, the index \lstinline{t} is bogus; that is,
it does not track the types of terms but remains constant everywhere.
Using the naive version of \lstinline{mprsi1} above, we can write an evaluator
similar to \lstinline{veval} for the untyped HOAS (\lstinline{Mu1 ExpF_u ()})
via the value domain (\lstinline{Mu1 V_u ()}), which would obviously
not terminate for some input.

Why did we believe that \lstinline{veval} always terminates?
Because it evaluates a well-typed HOAS, whose type is encoded as
an index \lstinline{t} in the recursive datatype \lstinline{(Exp t)}. That is,
the use of indices as types is the key to the termination property.
Therefore, our idea is to restrict the use of the abstract operations
by enforcing constraints over their indices; in that way, 
we would still be able write \lstinline{veval} for the typed HOAS,
but would get a type error when we try to write an evaluator for
the untyped HOAS.


We suggest that some of the abstract operations of \lstinline{mprsi1} should
only be applied to the abstract values whose indices are smaller in size
compared to the size of the argument index. For the \lstinline{veval} example,
we define being smaller as the structural ordering over types, that is,
\lstinline{t1 < (t1 -> t2)} and \lstinline{t2 < (t2 -> t1)}.
We have two candidates for the type signature of \lstinline{mprsi1}:
\begin{itemize}
\item Candidate 1: restrict uses of both \lstinline{cast} and \lstinline{uncast}
\begin{lstlisting}
mprsi1 :: (forall r j. (forall i. (i<j) => r i -> Mu1 f i)  --  cast
               -> (forall i. (i<j) => Mu1 f i -> r i)  --  uncast
               -> (forall i.          r i -> a i)      --  call
               -> (f r j -> a j)          ) -> Mu1 f i -> a i
\end{lstlisting}
\item Candidate 2: restrict the use of \lstinline{uncast} only
\begin{lstlisting}
mprsi1 :: (forall r j. (forall i.           r i -> Mu1 f i)  --  cast
               -> (forall i. (i<j) =>  Mu1 f i -> r i)  --  uncast
               -> (forall i.           r i -> a i)      --  call
               -> (f r j -> a j)          ) -> Mu1 f i -> a i
\end{lstlisting}
\end{itemize}
We strongly believe that the first candidate always terminates,
but it might be overly restrictive. Maybe the second candidate is
enough to guarantee termination? Both candidates allow defining
\lstinline{veval}, since one can define \lstinline{unVal}
using \lstinline{mprsi1} with either one of the candidates.
Both candidates forbid the definition of an evaluator over the untyped HOAS,
because neither supports extracting functions from the untyped value domain.

We need further studies to prove termination properties of \lstinline{mprsi}.
The sized-type approach, discussed in the related work section,
seems to be relevant to showing termination of \lstinline{mprsi}.
However, existing theories on sized-types are not directly applicable to
\lstinline{mprsi} because they are focused on positive datatypes, but
not negative datatypes.


\begin{figure}
\lstinputlisting[
	caption={Mendler-style parametric higher-order iteration
		(\mphit{}) at kind $*$ and $*\to*$.
		\label{lst:PRec}},
	firstline=4]{PRec.hs}

\lstinputlisting[
	caption={The \lstinline{showExp} example revisited using \mphit{*}.
		\label{lst:PHOASshow}},
	firstline=5,lastline=18]{PHOASshow.hs}

\lstinputlisting[
	caption={Desugaring let-expressions into applicatinos (\aka\ let-inlining) using \mphit{*}.
		\label{lst:PHOASdesug}},
	firstline=20]{PHOASshow.hs}
\end{figure}

\subsection{Mendler-style parametric higher-order iteration}
\label{sec:ongoing:mphit}
Inspired by the conventional style iteration over PHOAS \cite{BahHvi12}
(discussed in \S\ref{sec:relwork:PCDT}), we formulate its Mendler-style
counterpart \emph{Mendler-style parametric higher-order iteration} (\mphit{}).
Listing~\ref{lst:PRec} illsutrates Haskell transcription of \mphit{}
at kind $*$ and $*\to*$. Note that datatype definitions of $\hat\mu$ and
type signatures of \mphit{} at both kinds have virtually identical structure
except for the index $i$ and that the implementations of \mphit{*} and
\mphit{*\to*} have exactly the same structure.

\begin{figure}
\lstinputlisting[
	caption={The \lstinline{eval} example revisited using \mphit{*\to*}.
		\label{lst:PHOASeval}},
	firstline=4]{PHOASeval.hs}
\vspace*{-3ex}
\end{figure}

A notable difference from \msfit{},
besides the extra parameter (\lstinline{r_}) in base functors
(discussed in \S\ref{sec:relwork:PCDT}), is that \mphit{} provides only one
abstract operation (abstract recursive call) as you can observe from
the type synonym definitions of \lstinline{Phi0} and \lstinline{Phi1}.
For instance, \lstinline{(r a -> a)} is the type of
the abstract recursive call provided by \mphit{*}.
Recall that \msfit{} provides two abstract operations (abstract inverse
and recursive call) while \mit{} provides one (recursive call only).
As a result, first equations of \mphit{} in the definitions of \mphit{*} and
\mphit{*\to*} are exactly the same in structure as the definitions of \MIt{*}
and \MIt{*\to*} in Listing~\ref{lst:reccomb}. Hence, the revisited examples of
\lstinline{showExp} (Listing~\ref{lst:PHOASshow}) and \lstinline{eval}
(Listing~\ref{lst:PHOASshow}) become even more succinct than their corresponding
examples using \msfit{} (Listings~\ref{lst:HOASshow} and \ref{lst:HOASeval}),
simply omitting the uses of abstract inverse operations in the definitions
of \lstinline{phi} functions. We can view that uses of inverse operations
in \mphit{} are delegated to constructor functions of $\hat\mu$-types that
involves contravariant recursive occurrences; for instance, \lstinline{lam}
in Listing~\ref{lst:PHOASshow} is defined in terms of \lstinline{Var_0},
which is the constructor of \lstinline{Mu_0} for injecting inverses.

Bahr and Hvited \cite{BahHvi12} exemplified the strength of their PHOAS-based
iteration, compared to those \cite{FegShe96,bgb,AhnShe11} based on ordinary
(or strong) HOAS, by defining a desugaring function that transforms
let-expressions into applications (\aka\ let-inlining).
In Listing~\ref{lst:PHOASdesug}, we illustrate the same example
in the Mendler style. Note the simplicity of our Mendler-style version
(\lstinline{desugExp}) --- no need for class instances for functor,
difunctor, or higher-order versions of such algebras.

Although we transcribed \mphit{} in Haskell, we have not yet proved
its termination property. To prove its termination, we should find
an embedding of \mphit{} in a strongly normalizing calculus and study
the equational properties of that embedding, just as we did for \msfit{}
in \S\ref{sec:theory}. We think System \Fixw\ is a good candidate calculus
for trying to embed \mphit{} because we can use polarized kinds to ensure
that parameters \lstinline{r} and \lstinline{r_} are always used covariant
and contravariant positions, respectively, in base functor definitions.
Studying the relation between $\mu$ and $\hat\mu$, as we did for
$\mu$ and $\mu'$ in \S\ref{sec:murec}, is another subject of future work.

\section{Summary and Future Work}

We reviewed Mendler-style iteration (\MIt{}) and
primitive recursion (\mpr{}) with their typical examples,
the list length function (Listing~\ref{lst:Len}) and
the factorial function (Listing~\ref{lst:mprim}), respectively.
\mpr{} extends \mit{} with the additional cast operation that
converts abstract recursive values to concrete recursive values.
Moreover, we reviewed Mendler-style iteration with syntactic
inverses (\msfit{}) with the HOAS formatting example (Listing~\ref{lst:HOASshow});
this is the ``hello world'' example of recursion schemes over
mixed-variant datatypes. The abstract inverse operation provided
by \msfit{}, which is not present in \MIt{}, makes it useful over
mixed-variant datatypes.

We formulated the type-preserving evaluator for the simply-typed HOAS
(Listing~\ref{lst:HOASeval}). This evaluator demonstrates the usefulness
of \msfit{} over indexed mixed-variant datatypes. Moreover, this
example is a novel theoretic discovery that type-preserving HOAS
evaluation can directly (\ie, without translation into intermediate
data structure such as first-order syntax) embedded into System \Fw\ 
because we proved termination of the HOAS evaluator by embedding
\msfit{} into System \Fw\ (\S\ref{sec:theory:embed}). Moreover,
we studied the equational properties of the embedding
(\S\ref{sec:theory:embed}-\ref{sec:theory:srpair}) and
the subtype relation between ordinary fixpoint types for \MIt{} and
their corresponding inverse-augmented fixpoint types for \msfit{}
(\S\ref{sec:murec}).

We introduced the idea of Mender-style iteration with a sized index
(\mprsi{}) motivated by the example of type-preserving evaluation
via semantic domain (Listing~\ref{lst:HOASevalV}), in contrast to 
the evaluation example via native values of the host language
using \msfit{} (Listing~\ref{lst:HOASeval}). \mprsi{} extends \mpr{}
with the additional abstract uncast operation, which is
the inverse of the abstract cast operation provided by \mpr{} as well.
However, the uncast operation needs to be restricted in order to
guarantee termination. Termination proof for \mprsi{} needs
further investigation.

We introduced Mendler-style iteration over PHOAS (\mphit{}) and
demonstrated its usefulness by writing the type-preserving evaluator
over typed PHOAS (Listing~\ref{lst:PHOASeval}); this is similar to
the HOAS evaluator using \msfit{} (Listing~\ref{lst:HOASeval}) but
even more succinct because abstract inverses are not needed.
Moreover, we can write examples using \mphit{} that are
not expressible using \msfit{} such as the desugaring of
higher-order syntax (Listing~\ref{lst:PHOASdesug}). We hope to show
termination of \mphit{} by finding its embedding in System \Fixw,
which is an extension of System \Fw\ that can embed \mpr{}.

Mendler-style recursion schemes naturally extends term-indexed datatypes
(\eg, length-indexed lists) so that one can express more fine-grained
properties of programs in their types Ahn, Sheard, Fiore, and Pitts
\cite{AhnSheFioPit13} developed a term-indexed calculi System $\F_i$
by extending System \Fw\ with term indices in order to embed
Mendler-style recursion schemes such as \MIt{} and \msfit{} over
term-indexed datatypes. System $\textsf{Fix}_i$ \cite{Ahn14thesis} is
a similar extension to System \Fixw\ that can embed \mpr{} and
(hopefully) \mphit{} over term-indexed datatypes.

Based on the theories of term-indexed calculi, we are designing and
implementing a language called Nax, named after \emph{Nax P. Mendler},
that supports Mendler-style recursion schemes over
both type- and term-indexed datatypes as native language constructs.
The Nax language \cite{Ahn14thesis} is designed to adopt advantages
of both functional programming languages (\eg, mixed-variant datatypes,
type inference) and dependently-typed proof assistants (\eg,
fine-grained properties, logical consistency). Semantics of Nax can be
understood by embedding its key constructs such as datatypes and
recursion schemes into the term-indexed calculi.

One of the challenges in the language design is to choose as many
useful set of Mendler-style recursion schemes, including ones
for mixed-variant datatypes, that have compatible embeddings
in a term-indexed calculus. Not all recursion schemes would necessarily
have close relationship between their fixpoint types, as the subtyping
relation between fixpoints of \MIt{} and \msfit{} discussed
in \S\ref{sec:murec}. \MIt{} and \mpr{} are compatible as well.
However, we think it may be difficult to find compatible embeddings for
both \mpr{} and \msfit{}. We hope to discover an embedding of \mphit{}
that is compatible with the embedding of \mpr{}.

There are several other features in consideration to develop Nax to
become a more powerful and practical language. Some have already been
implemented and awaiting theoretical clarifications, while others
are just preliminary thoughts:
restrictive form of kind polymorphsim, pattern match coverage checking,
generalization of arrow (\ie, function) types in abstract operations
to generalize Mendler-style recursion schemes even further (\eg,
monadic recursion \cite{bahr11wgp}), and handling computations that
cannot (or need not) be internally proved terminating by the type system
(\eg, bar types \cite{constable1987partial}, mobile types \cite{CasSjoWei14}).


\section*{Acknowledgements}
%% \subparagraph*{Acknowledgements}
Thanks to Gabor Greif and Keewon Seo for their proofreading, and
Ralph Matthes for his suggestion on looking into interpolation.

%%% TODO
%% Syntactic consideration of recursive types. about LICS 96. Fiore.
%% related to mphit

%%% TODO
%% in the conclusion:
%% open problem need for theory/methods to expressiveness (difference)

%%% discuss in conclusion ????
%% mit vs msfit vs etc
\bibliography{main}

\vspace*{7.5ex}
\begin{lstlisting}[
	caption={An embedding algorithm in Haskell
		% for the constructors of
		% regular datatypes used with \lstinline{msfcata0}
		% transcribed in Haskell.
		\label{lst:embed}}]
data Sig = P | M -- %\textcomment{$P$ and $M$ stands for $+$ and $-$ %}

data Ty = TV Var | All Var Ty | Ty :-> Ty
        | Unit | Ty :*: Ty -- %\textcomment{These can be encoded too but%}
        | Void | Ty :+: Ty -- %\textcomment{for convenience of presentation.%}
        
data Tm = Fn (Tm -> Tm) | Tm :$ Tm -- %\textcomment{term representation using HOAS%}
        | Cunit | Tm :* Tm | Fst Tm | Snd Tm
        | L Tm | R Tm | CaseLR Tm (Tm -> Tm) (Tm -> Tm)
	| Lift Tm Tm | Phi -- %\textcomment{Added constants for%} lift %\textcomment{and%} phi.

type Var = Int -- %\textcomment{Give some suitable type for variable.%}

flipSig :: Sig -> Sig
flipSig M = P
flipSig P = M

-- %\textcomment{For a base structure $F$ defined as ~ \textbf{data}%} F r = C t1 ... tn | ... ,
-- %\textcomment{the embedding of the constructors of %} Rec0 F %\textcomment{ has the form of %}
-- %\textcomment{$c = \inR\;(\lambda \varphi.\varphi~\eta^{-1}~(C\;(\textit{rEm}\;t_1)\;\cdots\;(\textit{rEm}\;t_n)))))$ where \textit{rEm} is defined as below:%}
rEm :: Sig -> Var -> Ty -> Tm -> Tm
rEm _ _ Unit      = id -- %\textcomment{or %} const Cunit
rEm _ _ Void      = id
rEm _ r (TV x) | r/=x = id -- %\textcomment{Ignore variables other than the recursive one.%}
rEm P _ (TV _)   = Lift Phi -- %\textcomment{Apply%} lift phi %\textcomment{in positive occurrence%}
rEm M _ (TV _)   = L       -- %\textcomment{and apply $\inL$ in negative occurrence.%}
rEm p r (a :-> b) = \f -> Fn (\x -> rEm p r b (f :$ rEm p' r a x))
                  where p' = flipSig p
rEm p r (a :*: b) = \x -> rEm p r a x :* rEm p r b x
rEm p r (a :+: b) = \x -> CaseLR x (rEm p r a) (rEm p r b)
rEm p r (All x b) | r/=x = rEm p r b
rEm _ _ (All _ _) = error "should have been alpha renamed"
\end{lstlisting}

\newpage
\appendix
\section{Appendix:
	a type-directed embedding algorithm for the constructors of
	regular datatypes used with \lstinline{msfcata0}} \label{sec:appendix}

In \S\ref{sec:theory:embed}, we embedded the type constructors of
the untyped HOAS, which is a mixed-variant datatype with both positive
and negative occurrences, as annotated by $+$ and $-$
in $\textit{App}~r^{+}~r^{+}$ and $\textit{Abs}\;(r^{-} \to r^{+})$.
The HOAS example discussed in \S\ref{sec:theory:embed} has these
recursive occurrences either at topmost level, as in $\textit{App}~r^{+}~r^{+}$
occurring twice positively, or on both sides of the arrow type at topmost level,
as in $\textit{Abs}\;(r^{-} \to r^{+})$ occurring negatively
on the left-hand side and positively on the right-hand side.
Positive and negative occurrences are embedded differently --
recall that we used \lstinline{lift phi} for positive occurrences and $\inL$ for
negative occurrences (see p\pageref{align:embed} in \S\ref{sec:theory:embed}).

In general, recursive occurrences may occur more deeply inside
the type structure. For example, consider \lstinline{Rec0 F} where
\lstinline{data F r =  C}~$((r^{+} \to r^{-}) \to r^{+})$.
The leftmost occurrence of $r$ in the definition of $F$ is positive
because it is on the left hand side of the arrow at negative position
(negative of negative considered positive). Other type structures such as sums,
products, and universal quantifications do not have affect on the sign of
recursive occurrences in its subcomponents. That is, the subcomponents
maintain the same sign for recursive occurrences as their outermost position.

In Listing~\ref{lst:embed}, we describe an algorithm implementing the idea
discussed in the previous two paragraphs using Haskell. This algorithm is
type-directed, that is, it analyzes the given base structure $F$ to derive
the embeddings for the constructors of \lstinline{Rec0 F}. Here, we only
consider regular datatypes. By convention, the recursive argument $r$ always
comes at the last. For instance, the base structure for lists
\lstinline{data L a r = ...} where we take its fixpoint as \lstinline{Mu0(L a)}
for the list datatype. Therefore without loss of generality, we assume
that the base structures are defined as \lstinline{data F r = ...}.
Since our target calculus is polymorphic, we need variables (\textit{Var})
and universal quantification (\textit{All}) to represent types (\textit{Ty}).
We have sums (\lstinline{:+:}) and products (\lstinline{:*:}) and their
identities \lstinline{Unit} and \lstinline{Void} because base structures are
defined as sums of products of types. We can inline the embeddings of recursive
types of the form \lstinline{Rec0 G} occurring in the definition of $F$,
provided that $G$ is defined prior to $F$, because we already know
the embedding of \lstinline{Rec0} (see p\pageref{sec:theory:embed} in
\S\ref{sec:theory:embed}) and $G$ can also be embedded into System \Fw\ as
sums of products. Therefore, it suffice for the embedding function \textit{rEm}
in Listing~\ref{lst:embed} to analyze type structures (\textit{Ty})
in order to generate the embedded terms (\textit{Tm}).

We think it would be possible to prove the equational properties of
this type-directed embedding using interpolation, as used in the paper
by Matthes \cite{Matthes01a}. In fact, his paper has been a hint to derive
our algorithm in Listing~\ref{lst:embed}. Although we have only demonstrated
the algorithm for regular datatypes, we do not expect difficulties
in generalizing this algorithm to include non-regular datatypes, except for
truly nested datatypes
(e.g., \lstinline{data Bush a = BNil | BCons Bush (Bush a)}).
Embeddings for truly nested datatypes are going to be trickier than
the embeddings for the other datatypes because truly nested datatype
are indexed by their own types.



\end{document}



