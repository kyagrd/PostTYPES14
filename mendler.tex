\section{Mendler-style recursion schemes}
\label{sec:mendler}
In this section, we introduce basic concepts of two Mendler-style recursion
schemes: iteration (\lstinline{mcata}\,) and iteration with syntactic inverses
(\lstinline{msfcata}\,). Further details on Mendler-style recursion schemes,
including these two and more,
can be found in \cite{AhnShe11,AbeMatUus05,UusVen00,AbeMat04}.

\begin{figure}
\begin{lstlisting}[
	caption={Mendler-style iteration (\MIt{}) and
		Mendler-style iteration with syntactic inverses (\msfit{})
		at kind $*$ and $*\to *$ transcribed in Haskell
		\label{lst:reccomb}}]
data Mu0   (f::(* -> *))               = In0   (f (Mu0   f)  )
data Mu1 (f::(* -> *) -> (* -> *)) i = In1 (f (Mu1 f) i)

type a .-> b = forall i. a i -> b i
                      -- %\textcomment{call%}
type Phi0   f a = forall r. (r -> a) -> (f r -> a)
type Phi1 f a = forall r. (r .-> a) -> (f r .-> a)

mcata0   :: Phi0   f a ->  Mu0  f    -> a
mcata1 :: Phi1 f a ->  Mu1 f .-> a i

mcata0   phi (In0   x) = phi (mcata0   phi) x
mcata1 phi (In1 x) = phi (mcata1 phi) x
-- ------------------------------------------------------------
data Rec0   f a   = Roll0   (f (Rec0 f a)    ) | Inverse0   a
data Rec1 f a i = Roll1 (f (Rec1 f a) i) | Inverse1 (a i)
                      -- %\textcomment{inverse%}      -- %\textcomment{call%}
type Phi0'   f a = forall r. (a -> r a) -> (r a -> a) -> f (r a) -> a
type Phi1' f a = forall r. (a .-> r a) -> (r a .-> a) -> f (r a) .-> a

msfcata0   :: Phi0'   f a -> (forall a. Rec0   f a  ) -> a
msfcata1 :: Phi1' f a -> (forall a. Rec1 f a i) -> a i

msfcata0  phi r = msfcata phi r where
  msfcata phi (Roll0   x)      = phi Inverse0  (msfcata phi)  x
  msfcata phi (Inverse0 z)    = z

msfcata1  phi r = msfcata phi r  where
  msfcata  :: Phi1' f a -> Rec1 f a .-> a
  msfcata phi (Roll1 x)     = phi Inverse1 (msfcata phi)  x
  msfcata phi (Inverse1 z) = z
\end{lstlisting}

\textit{Note.}
The formulation of \lstinline{Rec1} and \lstinline{msfcata1} in
the previous work by Ahn and Sheard~\cite{AhnShe11} should be adjusted
as shown above. Although the previous formulation is type correct,
we realized that one cannot write useful examples over indexed datatypes
such as the type-preserving evaluator example in this paper. It was
an oversight due to the lack of testing their formulation
by examples over indexed mixed-variant datatypes.
\end{figure}

In Listing~\ref{lst:reccomb}, we illustrate the two recursion schemes,
\lstinline{mcata} and \lstinline{msfcata}, using Haskell.
We use a subset of Haskell, where we restrict the use of certain language
features and some of the definitions we introduce. We will explain
the details and motivation of these restrictions as we discuss
Listing~\ref{lst:reccomb}.

Each Mendler-style recursion scheme is described by a pair:
a type fixpoint (\eg, \lstinline{Mu0}, \lstinline{Rec0}) and
its constructors (\eg, \lstinline{In0}, \lstinline{Roll0}),
and the recursion scheme itself (\eg, \lstinline{mcata0}, \lstinline{msfcata0}).
A Mendler-style recursion scheme is characterized by
the abstract operations it supports. The types of
these abstract operations are evident in the type signature
of the recursion scheme. In Listing~\ref{lst:reccomb},
we emphasize this by factoring out the type of the first argument ($\varphi$)
as a type synonym prefixed by \lstinline{Phi}. Note the various synonyms
for each recursion scheme -- \lstinline{Phi0} has one abstract operation
and \lstinline{Phi0'} has two.

Mendler-style recursion schemes take two arguments.
The first is a function\footnote{By convention,
	we denote the function as $\varphi$. Which is why
	the type synonyms are prefixed by \lstinline{Phi}.}
that will be applied to concrete implementations of the abstract operators,
then uses these operations to describe the computation.
The second argument is a recursive value to compute over.
One programs by supplying specific instances of the first argument $\varphi$.

\subsection{Mendler-style iteration}
\label{sec:mendler:it}
Mendler-style iteration (\MIt{}) operates on recursive types constructed by
the fixpoint $\mu$. The fixpoint $\mu$ is indexed by a kind. We describe
$\mu$ at kind $*$ and $*\to*$ in Listing~\ref{lst:reccomb}.
We enforce two restrictions on the Haskell code in the Mendler style examples:
\begin{itemize}
\item Recursion is allowed only in the definition of the fixpoint at type-level,
and in the definition of the recursion scheme at term-level.
The type constructor \lstinline{Mu0} expects a non-recursive base structure
\lstinline{f :: * -> *} to construct a recursive type \lstinline{(Mu0 f :: *)}.
The type constructor \lstinline{Mu1} expects a non-recursive base structure
\lstinline{f :: (* -> *) -> (* -> *)} to construct a recursive type constructor
\lstinline{(Mu1 f :: * -> *)}, which expects one type index
(\lstinline{i :: *}). We do not use recursive datatype definitions (as
natively supported by Haskell) elsewhere. We do not use recursive function
definitions either, except to define Mendler-style recursion schemes.

\item Elimination of recursive values is only allowed via the recursion scheme.
One is allowed to freely introduce recursive values using \In{}-constructors,
but not allowed to freely eliminate (\ie, pattern match against \In{})
those recursive values. Note that \lstinline{mcata0} and \lstinline{mcata1}
are defined using pattern matching against \In{*} and \In{*\to*}.
Pattern matching against them elsewhere is prohibited.
\end{itemize}

The type synonyms \lstinline{Phi0} and \lstinline{Phi1} describe the types of
the first arguments of \lstinline{mcata0} and \lstinline{mcata1}.
These type synonyms indicate that Mendler-style iteration supports
one abstract operation: abstract recursive call \lstinline{(r -> a)}.
The type variable $r$ stands for an abstract recursive value, which could
be supplied to the abstract recursive call as an argument. Since $r$ is
universally quantified within \lstinline{Phi0} and \lstinline{Phi1},
functions of type \lstinline{Phi0 f a} and \lstinline{Phi1 f a} must be
parametric over $r$ (\ie, must not rely on examining any details of $r$-values).
In \lstinline{Phi0}, \lstinline{(r -> a)} is the type for
an abstract recursive call, which computes an answer of type \lstinline{a}
from the abstract recursive type \lstinline{r}.
This abstract recursive call is used to implement a function of type
\lstinline{f r -> a}, which computes an answer (\lstinline{a}) from
\lstinline{f}-structures filled with abstract recursive values (\lstinline{r}).
Similarly, \lstinline{(forall i.r i -> a i)} in \lstinline{Phi1} is the type
for an abstract recursive call, which is an index preserving function that
computes an indexed answer \lstinline{(a i)} from an indexed recursive value
\lstinline{(r i)}. In the Haskell definitions of \lstinline{mcata0} and
\lstinline{mcata1}, these abstract operations are made concrete by
a native recursive call. Note that the first arguments to
\lstinline{phi} in the definitions of \lstinline{mcata0} and \lstinline{mcata1}
are \lstinline{(mcata0 phi)} and \lstinline{(mcata1 phi)}.

Uses of Mendler-style recursion schemes are best explained by examples.
Listing~\ref{lst:Len} is a well-known example of a list length function
defined in terms of \lstinline{mcata0}. The recursive type for lists
\lstinline{(List  p)} is defined as a fixpoint of \lstinline{(L p)},
where \lstinline{L} is the base structure for lists. The data constructors
of \lstinline{List}, \lstinline{nil} and \lstinline{cons}, are defined
in terms of \lstinline{In0} and the data constructors of \lstinline{L}.
We define \lstinline{length} by applying \lstinline{mcata0} to
the \lstinline{phi} function. The function \lstinline{phi} is defined
by two equations, one for the \lstinline{N}-case and the other for
the \lstinline{C}-case. When the list is empty (\lstinline{N}-case),
the \lstinline{phi} function simply returns 0. When the list has an
element (\lstinline{C}-case), we first compute the length of the tail
(\ie, the list excluding the head, that is, the first element) by
applying the abstract recursive call \lstinline{(len :: r -> Int)}\footnote{
	Here, the answer type is \lstinline{Int}. }
to the (abstract) tail \lstinline{(xs :: r)},\footnote{
	Note that \lstinline{C x xs :: L p r} since \lstinline{xs :: r}.}
and, then, we add 1 to the length of the tail \lstinline{(len xs)}.

\begin{figure}
\lstinputlisting[
	caption={List length example using \lstinline{mcata0} \label{lst:Len}},
	firstline=3]{Len.hs}
\vspace*{-3ex}
\end{figure}


\subsection{Mendler-style iteration with syntactic inverses}
\label{sec:mendler:sf}
Mendler-style iteration with syntactic inverses (\msfit{}) operates on
recursive types constructed by the fixpoint $\mu'$. The fixpoint $\mu'$
is a non-standard fixpoint additionally parametrized by the answer type ($a$)
and has two constructors $\In{}'$ and \textit{Inverse}. $\In{}'$-constructors
are analogous to \In{}-constructors of $\mu$. \textit{Inverse}-constructors
hold answers to be computed by \msfit{}. For example,% \footnote{
	% In fact, this example is ill typed, because \msfit{} expects
	% its second argument type to be parametric over
	% (\ie, does not rely on specifics of) the answer type.
	% This example is just to illustrate the intuitive idea.}
the result of computing \lstinline{ msfcata  phi (Inverse0 5) }
is \lstinline{5} regardless of \lstinline{phi}.
The stylistic restrictions on the Haskell code involving \msfit{} are:
\begin{itemize}
\item Recursion is only allowed by the fixpoint at type-level ($\mu'$)
and by the recursion scheme (\msfit{}) at term-level.
We do not rely on recursive datatype definitions and function definitions
defined by the general recursion natively supported in Haskell.
\item Elimination of recursive values is allowed via the recursion scheme.
One is allowed to freely construct recursive values using $\In{}'$-constructors,
but not allowed to freely eliminate (\ie, pattern match against $\In{}'$) them.
Pattern matching against \textit{Inverse} is also forbidden.
\end{itemize}
These restrictions are similar to the stylistic restrictions involving \MIt{}.


The abstract operations supported by \msfit{} are evident
in the first argument type -- \lstinline{Phi0'} and \lstinline{Phi1'}
are the type synonyms for the first argument types of \lstinline{msfcata0}
and \lstinline{msfcata1}. Note that the abstract recursive type $r$ is also
additionally parametrized by the answer type $a$ in the type signatures
of \lstinline{msfcata0} and \lstinline{msfcata1}, since $\mu'$ is additionally
parametrized by $a$. In addition to the abstract recursive call, \msfit{}
also supports the abstract inverse operation. Note that the types for
abstract inverse (\lstinline{(a -> r a)} and \lstinline{(a i -> r a i)})
are indeed the types for inverse functions of abstract recursive call
(\lstinline{(r a -> a)} and \lstinline{(r a i -> a i)}). Instead of using
actual inverse functions to compute inverse images from answer values
during computation, one can hold intermediate answer values, whose inverse
images are irrelevant, inside \textit{Inverse}-constructors during
the computation using \msfit{}.

The type signature of \msfit{} expects the second argument to be
parametric over the answer type. Note the second argument types
\lstinline{(forall a. Rec0 f a)} and \lstinline{(forall a. Rec1 f a i)}
in the type signatures of \lstinline{msfcata0} and \lstinline{msfcata1}. 
Using \textit{Inverse} to construct recursive values elsewhere is, in a way,
prohibited due to the second argument type of \msfit{}. Using \textit{Inverse}
to construct concrete recursive values makes the answer type specific.
For example, \lstinline{(Inverse0 5) :: Rec0 f Int}, whose answer type
made specific to \lstinline{Int}, cannot be passed to \lstinline{msfcata0}
its second argument. The constructor \textit{Inverse} is only intended to
define \msfit{} and its first argument (\lstinline{phi}). One can indirectly
access \textit{Inverse} via the abstract inverse operation supported by
\msfit{}. Note, in the Haskell definitions of \lstinline{msfcata0} and
\lstinline{msfcata1}, the second arguments to \lstinline{phi} are
\lstinline{Inverse0} and \lstinline{Inverse1}. That is, the abstract inverse
operation is implemented by the \textit{Inverse}-constructor.

\begin{figure}
\lstinputlisting[
	caption={Formatting an untyped HOAS expression into a \lstinline{String}
		\label{lst:HOASshow} (adopted from \cite{AhnShe11}) },
	firstline=5]{HOASshow.hs}
\vspace*{-4ex}
\end{figure}

The HOAS formatting is a ``hello world'' example repeatedly formulated
in studies on recursion schemes over HOAS; \eg, \cite{FegShe96,bgb,BahHvi12}
to mention a few in the conventional style. This example is interesting because it
is a simplification of a recurring pattern (or functional pearl \cite{AxeCla13})
of conversion from higher-order syntax to first-order syntax, which is often
found in implementations of embedded domain specific languages.
Listing~\ref{lst:HOASshow} illustrates a Mendler-style formulation
(\lstinline{showExp}) of this example using \msfit{}.


The key characteristic of \lstinline{showExp} is apparent in the user-defined
combining function \lstinline{phi}. From the type of \lstinline{phi}, we know
that the result of iteration over a HOAS term \lstinline{e} is a function;
more specifically,
\lstinline{msfcata0  phi e :: [String ] -> String}. An infinite list of
fresh variable names (\lstinline{vars})\footnote{
	To be strictly complacent to the conventions of the Mendler style,
	we would have to formulate a co-recursive datatype
	that generates infinite list of variable names.
	Here, we simply use Haskell's lazy lists because our focus here is
	not co-recursion but introducing an example using \msfit{}.}
is supplied as an argument to \,\lstinline{msfcata0  phi e}\, to obtain
a string that represents \lstinline{e}.

Definition of \lstinline{phi} consists of two equations.
The first equation for \lstinline{App} is a typical structural recursion
over positive occurrences of recursive subcomponents. The second equation
for \lstinline{Lam} exploits the abstract inverse
(\lstinline{inv :: ([String ] -> String) -> r ([String ] -> String)})
provided by \msfit{} to handle the negative recursive occurrence.
When formatting a \lstinline{Lam}-expression, one should supply a fresh variable
to represent the bounded variable (which is the negative recursive occurrence)
introduced by \lstinline{Lam}. Here, we consume one fresh name from the supplied
list of fresh names by pattern matching \lstinline{(v:vs)}, and take an inverse
of a constant function that will return the name \lstinline{(inv(const  v))},
which has an appropriate type to pass into the function \lstinline{z} contained
in constructor \lstinline{Lam}. Since the result of this application
\lstinline{z(inv(const  v))} corresponds to a positive recursive occurrence,
we simply apply the abstract recursive call \lstinline{show'}.

