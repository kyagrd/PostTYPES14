\section{$\pmb{\mu'}$-fixpoint is a subtype of $\pmb{\mu}$-fixpoint}\label{sec:murec}
We discussed the usefulness of \msfit{} by the illustrating examples on HOAS.
If one is to design a language based on Mendler-style recursion schemes,
one would want to support as many useful recursion schemes available,
including \MIt{} and \msfit{}. One issue in such design is that we have
two different fixpoints $\mu$ and $\mu'$. The standard fixpoint $\mu$
does not come with syntactic inverses while $\mu'$ comes with
its syntactic inverse. It would be a bad design choice to provide 
two unrelated fixpoints and let users deal with them manually.
We would like to apply as many recursion schemes to one recursive value
without manual conversion.

\begin{figure}
\lstinputlisting[
	caption={Coercion from $\mu'$-values to $\mu$-values
		using \lstinline{msfcata0} \label{lst:rec2mu} },
	firstline=4]{Exp2Expr.hs}
\vspace*{-3ex}
\end{figure}

We discovered that $\mu'$ is a subtype of $\mu$. Listing\;\ref{lst:rec2mu}
illustrates a mapping from \lstinline{(forall a.Rec0 ExpF  a)}
to \lstinline{(Mu0 ExpF)} implemented using \lstinline{msfcata0},
where \lstinline{ExpF} is a base structure for the untyped HOAS.
Since we have two fixpoints, \lstinline{Rec0} and \lstinline{Mu0},
we can define two recursive datatypes from the base structure \lstinline{ExpF}.
One is \lstinline{Exp} defined as \lstinline{(forall a.Rec0 ExpF  a)} and
the other is \lstinline{Expr} defined as \lstinline{Mu0 ExpF}.
The function \lstinline{exp2expr  :: Exp -> Expr}\, implements the mapping from
\lstinline{Rec0}-based HOAS expressions to \lstinline{Mu0}-based
HOAS expressions. Note, \lstinline{exp2expr}\, is defined
using \lstinline{msfcata0}.  Since there exists an embedding of
\lstinline{Mu0} and \lstinline{msfcata0} into System~\Fw\ \cite{AhnShe11},
\lstinline{exp2expr}\, is admissible in System~\Fw. However, it is not likley
that we can embed a coercion function for an arbitrary base structure $f$,
\lstinline{mu2rec :: (forall a.Rec0 f a) -> Mu0 f}, in System~\Fw.\footnote{
	The discussions in \S\ref{sec:theory} on the embedding of \msfit{}
	suggests why the \lstinline{mu2rec} is not likely to be embedded
	in System~\Fw, but its specific instances, such as
	\lstinline{exp2expr}, can be embedded in System~\Fw.}
We conjecture that it should be possible to derive a coercion function
from $\mu'$-values to $\mu$-values when given a specific instance of
the base structure. Therefore, when designing a language based on Mendler-style
recursion schemes, we may support coercion from $\mu'$-values to $\mu$-values.

Coercion the other way around, from $\mu$-values to $\mu'$-values,
is not likely to be possible in general, but might be possible
only when the answer type of the $\mu'$-values (\eg, \lstinline{a}
in \lstinline{Rec0 ExpF  a}) has been monomorphically instantiated
to the final answer value. That is, we attempted to convert from
\lstinline{Exp' Expr} to \lstinline{Expr}, rather than from \lstinline{Exp}
(\ie, \lstinline{forall a.Exp' a}) to \lstinline{Expr}.\footnote{
	Also note that \lstinline{a} in \lstinline{(Rec0 ExpF  a)}
	in the type signature of $\msfit{}'$
	is not quantified. {c.f.} (\lstinline{(forall a.Rec0 f a)}
	in the type signature of $\msfit{*}$.}
We illustrate this idea in Listing~\ref{lst:mu2rec}, which is still
an incomplete attempt since there is no termination guarantee for
\lstinline{expr2exp'}. Note that \lstinline{expr2exp'} is not defined
using a Mendler-style recursion scheme but using general recursion.

\begin{figure}
\begin{lstlisting}[
	caption={An incomplete attempt to convert from $\mu$-values
		to $\mu'$-values \label{lst:mu2rec} } ]
msfcata'  :: Phi0' f a -> Rec0 f a -> a
msfcata' phi (Roll0 x)     = phi Inverse0 (msfcata' phi)  x
msfcata' phi (Inverse0 z) = z

exp'2expr :: Exp' Expr -> Expr  -- %\textcomment{i.e, %} Rec0 ExpF (Mu0 ExpF) -> Mu0 ExpF
exp'2expr = msfcata' phi where
  phi inv p2r (Lam f)    = In0(Lam((\x -> p2r (f (inv x)))))
  phi inv p2r (App e1 e2)  = In0(App (p2r e1) (p2r e2))

expr2exp' :: Expr -> Exp' Expr  -- %\textcomment{i.e., %} Mu0 ExpF -> Rec0 ExpF (Mu0 ExpF)
expr2exp' (In0(Lam f))   = Roll0 (Lam (\x->expr2exp' (f (exp'2expr x))))
expr2exp' (In0(App e1 e2)) = Roll0 (App (expr2exp' e1) (expr2exp' e2))
\end{lstlisting}
\vspace*{-3ex}
\end{figure}

The coercion from \lstinline{(forall a.Rec0 ExpF  a)}
to \lstinline{(Mu0 ExpF)} exists. We believe that \lstinline{msfcata0} can
express more functions than \lstinline{mcata0} (e.g., \lstinline{showExp}
in Listing\;\ref{lst:HOASshow}). Then, it may be the case that the set of
values of \lstinline{(forall a.Rec0 f a)} is in fact more restrictive than
the set of values of \lstinline{(Mu0 f)}. The additional expressiveness of
\lstinline{msfcata0} may be a compensation for the restrictions on
the value of \lstinline{(forall a.Rec0 f a)}. In summary, 
\lstinline{(forall a.Rec0 f a)} is a subset of \lstinline{(Mu0 f)}.
We believe that this generalizes to arbitrary kinds other than $*$.

