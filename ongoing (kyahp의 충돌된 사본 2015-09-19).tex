% \section{Further work}
\section{Ongoing work}
\label{sec:ongoing}
We present two threads of ongoing work regarding
Mendler-style recursion schemes over mixed-variant datatypes ---
Mendler-style primitive recursion with a sized index (\S\ref{sec:ongoing:mprsi})
and
Mendler-style parametric iteration (\S\ref{sec:ongoing:mphit}).\vspace*{-1.5ex}

\subsection{Mendler-style primitive recursion with a sized index}
\label{sec:ongoing:mprsi}

In \S\ref{sec:mendler} and \S\ref{sec:HOASeval}, we discussed
Mendler-style iteration with a syntactic inverse, \lstinline{msfcata},
which is particularly useful for defining functions over
negative (or mixed-variant) datatypes. We demonstrated the usefulness of
\lstinline{msfcata} by defining functions over HOAS:
\begin{itemize}
\item the string formatting function \lstinline{showExp} for
	the untyped HOAS using \lstinline{msfcata0}
	(Listing\;\ref{lst:HOASshow}),% and
\item the type-preserving evaluator \lstinline{eval} for
	the simply-typed HOAS using \lstinline{msfcata1}
	(Listing\;\ref{lst:HOASeval}).\vspace*{-.7ex}
\end{itemize}

In this subsection, we speculate about another Mendler-style recursion scheme,
\lstinline{mprsi}, motivated by an example similar to the \lstinline{eval}
function. The name \lstinline{mprsi} stands for
Mendler-style primitive recursion with a sized index.\vspace*{.7ex}

\lstinputlisting[
	caption={A simply-typed HOAS evaluation via a user-defined value domain.
		\label{lst:HOASevalV}
		},
	firstline=4]{HOASevalV.hs}\vspace*{-1ex}

We review the \lstinline{eval} example and then compare
it to our motivating example \lstinline{veval} for \lstinline{mprsi}.
Both \lstinline{eval} and \lstinline{veval} are illustrated
in Listing\;\ref{lst:HOASevalV}. Recall that this code is written in Haskell,
following the Mendler-style conventions.
The function \lstinline{eval :: Exp t -> Id t} is
a type preserving evaluator that evaluates an HOAS expression of type
\lstinline{t} to a (Haskell) value of type \lstinline{t}.
The \lstinline{eval} function always terminates because
\lstinline{msfcata1} always terminates. Recall that \lstinline{msfcata1}
and \lstinline{Rec1} can be embedded into System~\Fw.

The motivating example \lstinline{veval :: Exp t -> Val t} is also 
a type-preserving evaluator. Unlike \lstinline{eval}, it evaluates to 
a user-defined value domain \lstinline{Val} of type \lstinline{t} (rather
than a Haskell value). The definition of \lstinline{veval} is similar to
\lstinline{eval}; both of them are defined using \lstinline{msfcata1}.
The first equation of \lstinline{phi} for evaluating
the \lstinline{Lam}-expression is essentially the same as
the corresponding equation in the definition of \lstinline{eval}.
The second equation of \lstinline{phi} for evaluating
the \lstinline{App}-expression is also similar in structure to
the corresponding equation in the definition of \lstinline{eval}.
However, the use of \lstinline{unVal} is problematic. In particular,
the definition of \lstinline{unVal} relies on pattern matching against
\lstinline{In1}. Recall that one cannot freely pattern match against
a recursive value in the Mendler style. Recursive values must be analyzed
(or eliminated) by using Mendler-style recursion schemes. It is not a problem
to use \lstinline{unId} in the definition of \lstinline{eval} because
\lstinline{Id} is non-recursive.

It is unlikely that \lstinline{unVal} can be defined using any of
the existing Mendler-style recursion schemes. So, we designed
a new Mendler-style recursion scheme that can express \lstinline{unVal}.
The new recursion scheme \lstinline{mprsi} extends \lstinline{mprim} with
an additional uncast operation. Recall that \lstinline{mprim} has
two abstract operations, call and cast. So, \lstinline{mprsi} has
three abstract operations, call, cast, and uncast. In the following paragraphs,
we explain the design of \lstinline{mprsi} step-by-step.

Let us try to define \lstinline{unVal} using \lstinline{mprim1} and examine
where it falls short: \lstinline{mprim1} provides two abstract operations,
\lstinline{cast} and {call}, as it can be seen from the type signature:
\begin{lstlisting}
mprim1 :: (forall r i. (forall i. r i -> Mu1 f i)  -- cast
              -> (forall i. r i -> a i)      -- call
              -> (f r i -> a i)        ) -> Mu1 f i -> a i
\end{lstlisting}
We attempt to define \lstinline{unVal} using \lstinline{mprim1} as follows:
\begin{lstlisting}
unVal :: Mu1 V (t1 -> t2) -> (Mu1 V t1 -> Mu1 V t2)
unVal = mprim1 phi where
  phi cast call (VFun f) = ...
\end{lstlisting}\vspace*{-1.3ex}
Inside the \lstinline{phi} function, we have a function
\lstinline{f :: (r t1 -> r t2)} over abstract recursive values.
We need to cast \lstinline{f} into a function over concrete recursive values
\lstinline{(Mu1 V t1 -> Mu1 V t2)}.
We should not need to use \lstinline{call}, since we do not expect
to use any recursion to define \lstinline{unVal}.
So, the only available operation is
\lstinline{cast :: (forall i.r i -> Mu1 f i)}.
Composing \lstinline{cast} with \lstinline{f}, we can get
\lstinline{(cast . f) :: (r t1 -> Mu1 V t2)}, whose codomain
\lstinline{(Mu1 V t2)} is exactly what we want. But, the domain
is still abstract \lstinline{(r t1)} rather than being concrete
\lstinline{(Mu1 V t1)}. We are stuck.

What additional abstract operation would help us complete
the definition of \lstinline{unVal}? We need an abstract operation
to cast from \lstinline{(r t1)} to \lstinline{(Mu1 V t1)}
in a contravariant position. If we had an inverse of cast,
\lstinline{uncast :: (forall i.Mu1 f i -> r i)}, we can
complete the definition of \lstinline{unVal} by composing
\lstinline{uncast}, \lstinline{f}, and \lstinline{cast}.
That is, \lstinline{uncast . f . cast :: (Mu1 V t1 -> Mu1 V t2)}.
Thus, we can formulate \lstinline{mprsi1} with a naive type signature
as follows:
\begin{lstlisting}
mprsi1 :: (forall r i. (forall i. r i -> Mu1 f i)  -- cast
               -> (forall i. Mu1 f i -> r i)  -- uncast
               -> (forall i. r i -> a i)      -- call
               -> (f r i -> a i)        ) -> Mu1 f i -> a i
mprsi1 phi (In1 x) = phi id id (mprsi1 phi) x
\end{lstlisting}\vspace*{-1.2ex}
Although the type signature above is type-correct, it is too powerful.
The Mendler-style uses types to forbid non-terminating computations
as ill-typed. Having both \lstinline{cast} and \lstinline{uncast} supports
the same ability as freely pattern matching over recursive values,
which can lead to non-termination. To recover the guarantee of termination,
we need to restrict the use of either \lstinline{cast} or \lstinline{uncast},
or both.

Let us see how this non-termination might occur. If we allowed
\lstinline{mprsi1} with the naive type signature above, we could write
an evaluator (similar to \lstinline{veval} but for an untyped HOAS),
which does not always terminate. This evaluator would diverge for terms
with self application.
Here, we walk through the process of defining an untyped HOAS.
The base structures of the untyped HOAS and its value domain
can be defined as follows:
\begin{lstlisting}
data ExpF_u r t = Lam_u (r t -> r t) | App_u (r t) (r t)
data V_u r t = VFun_u (r t -> r t)
\end{lstlisting}
Fixpoints of the structures above represent the untyped HOAS and
its value domain. Here, the index \lstinline{t} is bogus; that is,
it does not track the types of terms but remains constant everywhere.
Using the naive version of \lstinline{mprsi1} above, we can write an evaluator
similar to \lstinline{veval} for the untyped HOAS (\lstinline{Mu1 ExpF_u ()})
via the value domain (\lstinline{Mu1 V_u ()}), which would obviously
not terminate for some inputs.

Why did we believe that \lstinline{veval} always terminates?
Because it evaluates a well-typed HOAS, whose type is encoded as
an index \lstinline{t} in the recursive datatype \lstinline{(Exp t)}. That is,
the use of indices as types is the key to the termination property.
Therefore, our idea is to restrict the use of the abstract operations
by enforcing constraints over their indices; in that way, 
we would still be able to write \lstinline{veval} for the typed HOAS,
but would get a type error when we try to write an evaluator for
the untyped HOAS.


We suggest that some of the abstract operations of \lstinline{mprsi1} should
only be applied to the abstract values whose indices are smaller in size
compared to the size of the argument index. For the \lstinline{veval} example,
the structural ordering over types can be given as
\lstinline{t1 < (t1 -> t2)} and \lstinline{t2 < (t2 -> t1)}.
We have two candidates for the type signature of \lstinline{mprsi1}:
\begin{itemize}
\item Candidate 1: restrict uses of both \lstinline{cast} and \lstinline{uncast}
\begin{lstlisting}
mprsi1 :: (forall r j. (forall i. (i<j) => r i -> Mu1 f i)  --  cast
               -> (forall i. (i<j) => Mu1 f i -> r i)  --  uncast
               -> (forall i.          r i -> a i)      --  call
               -> (f r j -> a j)          ) -> Mu1 f i -> a i
\end{lstlisting}
\item Candidate 2: restrict the use of \lstinline{uncast} only
\begin{lstlisting}
mprsi1 :: (forall r j. (forall i.           r i -> Mu1 f i)  --  cast
               -> (forall i. (i<j) =>  Mu1 f i -> r i)  --  uncast
               -> (forall i.           r i -> a i)      --  call
               -> (f r j -> a j)          ) -> Mu1 f i -> a i
\end{lstlisting}
\end{itemize}
We strongly believe that the first candidate always terminates,
but it might be overly restrictive. Maybe the second candidate is
enough to guarantee termination? Both candidates allow defining
\lstinline{veval}, since one can define \lstinline{unVal}
using \lstinline{mprsi1} with either one of the candidates.
Both candidates forbid the definition of an evaluator over the untyped HOAS,
because neither supports extracting functions from the untyped value domain.

We need further studies to prove termination properties of \lstinline{mprsi}.
The sized-type approach, discussed in the related work section,
seems to be relevant to showing termination of \lstinline{mprsi}.
However, existing theories on sized-types are not directly applicable to
\lstinline{mprsi} because they are focused on positive datatypes, but
not negative datatypes.
\newpage
\lstinputlisting[
	caption={Mendler-style parametric iteration
		(\mphit{}) at kind $*$ and $*\to*$.
		\label{lst:PRec}},
	firstline=4]{PRec.hs}
%% \lstinputlisting[
%% 	caption={The \lstinline{showExp} example revisited using \mphit{*}.
%% 		\label{lst:PHOASshow}},
%% 	firstline=5,lastline=18]{PHOASshow.hs}
\lstinputlisting[
	caption={The \lstinline{eval} example revisited using \mphit{*\to*}.
		\label{lst:PHOASeval}},
	firstline=4]{PHOASeval.hs}

\subsection{Mendler-style parametric iteration}
\label{sec:ongoing:mphit}
Inspired by the conventional style iteration over PHOAS \cite{BahHvi12}
(discussed in \S\ref{sec:relwork:PCDT}), we formulate its Mendler-style
counterpart \emph{Mendler-style parametric iteration} (\mphit{}).
Listing~\ref{lst:PRec} illustrates Haskell transcription of \mphit{}
at kind $*$ and $*\to*$. Note that datatype definitions of $\hat\mu$ and
type signatures of \mphit{} at both kinds have virtually identical structure
except for the index $i$ and that the implementations of \mphit{*} and
\mphit{*\to*} have exactly the same structure.

A notable difference from \msfit{},
besides the extra parameter (\lstinline{r_}) in base functors
(discussed in \S\ref{sec:relwork:PCDT}), is that \mphit{} provides only one
abstract operation (abstract recursive call) as you can observe from
the type synonym definitions of \lstinline{Phi0} and \lstinline{Phi1}.
For instance, \lstinline{(r a -> a)} is the type of
the abstract recursive call provided by \mphit{*}.
Recall that \msfit{} provides two abstract operations (abstract inverse
and recursive call) while \MIt{} provides one (recursive call only).
As a result, the first equations of \mphit{} in the definitions of \mphit{*}
and \mphit{*\to*} are exactly the same in structure as the definitions of
\MIt{*} and \MIt{*\to*} in Listing~\ref{lst:reccomb}. Hence, the revisited
example of \lstinline{eval} (Listing~\ref{lst:PHOASeval}) is more succinct
than its corresponding example using \msfit{} (Listings~\ref{lst:HOASeval}),
omitting \textit{inv} in the definition of the \lstinline{phi} functions.
Note that the uses of \textit{inv} in \mphit{} are delegated to
the constructor functions of $\hat\mu$-types, which involve
contravariant recursive occurrences; for instance, \lstinline{lam}
in Listing~\ref{lst:PHOASeval} is defined in terms of \lstinline{Var_1},
which is the constructor of \lstinline{Mu_1} for injecting inverses.

Bahr and Hvited \cite{BahHvi12} exemplified the strength of their PHOAS-based
iteration, compared to those \cite{FegShe96,bgb,AhnShe11} based on ordinary
(or strong) HOAS, by defining a constant folding over a small language.
In Listing~\ref{lst:PHOAScfold}, we illustrate the constant folding example
in the Mendler style. Note the simplicity of our Mendler-style version of
constant folding --- no need for class instances for functor,
difunctor, or higher-order versions of such algebras (if one wants a
type-indexed version).
%%% TODO TODO TODO
\textcolor{red}{TODO give an intuitive argument
	why the use of unIn is safe here}

\begin{figure}
\lstinputlisting[
	caption={Constant folding using \mphit{*}. \label{lst:PHOAScfold}},
	firstline=219, lastline=233]{Rec2.hs}
\end{figure}

Although we transcribed \mphit{} in Haskell, we have not yet proved
its termination property (neither did Bahr and Hvited \cite{BahHvi12}
for their conventional version). To prove its termination, we should find
an embedding of \mphit{} in a strongly normalizing calculus and study
the equational properties of that embedding, just as we did for \msfit{}
in \S\ref{sec:theory}. We think System \Fixw\ is a good candidate calculus
for trying to embed \mphit{} because we can use polarized kinds to ensure
that parameters \lstinline{r} and \lstinline{r_} are always used covariant
and contravariant positions, respectively, in base functor definitions.
Studying the relation between $\mu$ and $\hat\mu$, as we did for
$\mu$ and $\mu'$ in \S\ref{sec:murec}, is another subject of future work.
%%%%
\textcolor{red}{TODO mention polarized kinds whie discussing this
(has it been introduced before in the related work?)}
\textcolor{red}{TODO also discuss Abel's $\hat{F}_\omega$ or
	something for sized types ???}
