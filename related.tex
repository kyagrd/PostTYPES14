\section{Related Work}\label{sec:relwork}
Here, we discuss several related work. First, we introduce Mendler-style
primitive recursion (\mpr{}) in \S\ref{sec:relwork:mpr} to lead up
the discussion of \mprsi{} (\S\ref{sec:ongoing:mprsi}). Next, we summarize
type-based termination and sized-type approach (in relation to \mpr{})
in \S\ref{sec:relwork:sized}. Lastly, we discuss a generic programming
library in Haskell that supports binders using parametric HOAS
in \S\ref{sec:relwork:PCDT}, which leads up the discussion of
\mphit{} (\S\ref{sec:ongoing:mphit}).

\subsection{Mendler-style primitive recursion}\label{sec:relwork:mpr}
\begin{figure}
\lstinputlisting[
	caption={Examples using Mendler-style primitive recursion
		\mpr{} at kind $*$\,: a constant time \lstinline{pred} and
		a \lstinline{factorial} function. \label{lst:mprim}},
        firstline=12]{Fac.hs}
\vspace*{-3ex}
\end{figure}

Termination of the Mendler-style iteration (\MIt{}) can be proved by embedding
\MIt{} into System \Fw\ as discussed in \S\ref{sec:mendler:it}. The embedding
of \MIt{} in \S\ref{sec:mendler:it} is \emph{reduction preserving}, that is,
the number of reduction steps using the embedding and using
the equational specification should differ no more than constant time factor.
Reduction preserving embedding of primitive recursion into System \Fw\ cannot
exist because it is known that ``induction is not derivable in second order
dependent type theory'' \cite{Geuvers01} and that ``primitive recursion
can be seen as the computational interpretation of induction through
the Curry-Howard interpretation of propositions-as-types'' \cite{Hallnas92}.
Although it is possible to simulate primitive recursion in terms of iteration,
it may become computationally inefficient. For example, \lstinline{pred}\,
in Listing~\ref{lst:mprim} could be defined using \MIt{} but its time complexity
would be at least linear to the input rather than constant time. Constant time
\lstinline{pred} is definable due to the abstract cast operation provided by
\mpr{}. This operation casts an abstract recursive values of type $r$ into
a concrete recursive values of type \lstinline{Mu0 f}\,; its type
\lstinline{(r -> Mu0 f)} is apparent from the type signature of
\lstinline{mprim0}. This cast operation computes in constant time because
it is implemented as the identity function (\lstinline{id}) in the definition
of \lstinline{mprim0}. A representative example of \mpr{} that actually use
recursion is a factorial function. The multiplication function \lstinline{times}
used in the definition of \lstinline{factorial} can be defined in terms of
\MIt{*} and an addition function, in turn, the addition function can be defined
in terms of \MIt{*} as well. Mendler-style primitive recursion generalize to
higher kinds in the same manner as \MIt{} and \msfit{}
(see Listing~\ref{lst:reccomb} in \S\ref{sec:mendler}).

Abel and Matthes \cite{AbeMat04} discovered a reduction preserving
embedding of the Mendler-style primitive recursion in System \Fixw,
which is a strongly normalizing calculus extended from System \Fw\ with
polarized kinds and equi-recursive fixpoints. Polarized kinds extend
the kind arrow with polarities in the form of $p\kappa_1 \to \kappa_2$
where polarity $p$ is either $+$, $-$, or $0$; meaning that the argument
must be used in positive, negative, or any position, respectively.
For example, in a polarized system, base structure \lstinline{N :: * -> *} 
for natural numbers in Listing~\ref{lst:mprim} could be assigned $+*\to*$
because its argument $r$ is only used covariantly, and, base structure
\lstinline{ExpF  :: * -> *} in Listing~\ref{lst:HOASshow} for the untyped HOAS
(see \S\ref{sec:mendler:sf}) must be assigned kind $0* \to *$ because
its argument $r$ is used in both covariant and contravariant positions.
The equi-recursive fixpoint $\textsf{fix}_\kappa : +\kappa \to \kappa$
in System~\Fixw\ can be applied only to positive base structures.\footnote{
	Otherwise, equi-recursive types are able to express
	diverging computation when they are not restricted to positive polarity}
Abel and Matthes encoded iso-recursive fixpoint operator $\mu$ in terms of
equi-recursive fixpoint operator \textsf{fix} by converting base structures of
arbitrary polarities into base structures of positive polarities, in order
to embed \mpr{} into System \Fixw.

\subsection{Type-based termination and sized types}\label{sec:relwork:sized}
\emph{Type-based termination} (coined by Barthe and others \cite{BartheFGPU04})
stands for approaches that integrate termination into type checking,
as opposed to syntactic approaches that reason about termination over
untyped term structures. The Mendler-style approach is, of course,
type-based. In fact, the idea of type-based termination was inspired by
Mendler \cite{Mendler87,Mendler91}. In the Mendler style, we know that
well-typed functions defined using Mendler-style recursion schemes always
terminate. This guarantee follows from the design of the recursion scheme,
where the use of higher-rank polymorphic types in the abstract operations
enforce the invariants necessary for termination.

Abel \cite{abel06phd,Abel12talkFICS} summarizes the advantages of
type-based termination as:
\emph{communication} (programmers think using types),
\emph{certification} (types are machine-checkable certificates),
\emph{a simple theoretical justification}
        (no additional complication for termination other than type checking),
\emph{orthogonality} (only small parts of the language are affected,
        \eg, principled recursion schemes instead of general recursion),
\emph{robustness} (type system extensions are less likely to
                        disrupt termination checking),
\emph{compositionality}
        (one needs only types, not the code, for checking the termination), and
\emph{higher-order functions and higher-kinded datatypes}
        (works well even for higher-order functions and non-regular datatypes).
In his dissertation \cite{abel06phd} (Section 4.4) on sized types,
Abel views the Mendler-style approach as enforcing size restrictions
using higher-rank polymorphism as follows:
\begin{itemize}
\item The abstract recursive type $r$ in the Mendler style corresponds to
        $\mu^\alpha F$ in his sized-type system (System \Fwhat),
        where the sized type
        for the value being passed in corresponds to $\mu^{\alpha+1} F$.
\item The concrete recursive type $\mu F$ in the Mendler style corresponds to
        $\mu^\infty F$ since there is no size restriction.
\item By subtyping, a type with a smaller size-index can be cast to
        the same type with a larger size-index.
\end{itemize}
The same intuition holds for the termination behaviors
of Mendler-style recursion schemes over positive datatypes.
For positive datatypes, Mendler-style recursion schemes terminate
because $r$-values are direct subcomponents of the value being eliminated.
They are always smaller than the value being passed in.
Types enforce that recursive calls are only well-typed,
when applied to smaller subcomponents.

Abel's System \Fwhat\ can express primitive recursion quite naturally
using subtyping. The casting operation $(r \to \mu F)$ in Mendler-style
primitive recursion corresponds to an implicit conversion by subtyping
from $\mu^\alpha F$ to $\mu^\infty F$ because $\alpha \leq \infty$.
System \Fwhat\ \cite{abel06phd} is closely related to
System \Fixw\ \cite{AbeMat04}. Both of these systems are base on
equi-recursive fixpoint types over positive base structures.
Both of these systems are able to embed (or simulate) Mendler-style
primitive recursion (which is based on iso-recursive types) via
the encoding \cite{Geu92} of arbitrary base structures into
positive base structures.

Abel's sized-type approach evidences good intuition concerning the reasons
that certain recursion schemes terminate over positive datatypes.
But, we have not gained a useful intuition of whether or not those
recursion schemes would terminate for negative datatypes, unless there is
an encoding that can translate negative datatypes into positive datatypes.
For primitive recursion, this is possible (as we mentioned above). However,
for our recursion scheme \MsfIt, which is especially useful over negative
datatypes, we do not know of an encoding that can map the inverse-augmented
fixpoints into positive fixpoints. So, it is not clear whether the sized-type
approach based on positive equi-recursive fixpoints can provide
a good explanation for the termination of \MsfIt.

In \S\ref{sec:ongoing:mprsi}, we will discuss
another Mendler-style recursion scheme (\mprsi{}), which is also useful over
negative datatypes and believed to have a termination property (not yet proved)
based on the size of the index in the datatype.


\subsection{Parametric compositional data types}
\label{sec:relwork:PCDT}
Bahr and Hvited developed a generic programming library in Haskell,
\emph{compositional data types} (CDT) \cite{bahr11wgp}, which builds on
Wouter Swierstra's ideas of \emph{data types \`a la carte} \cite{WSout08jfp}.
Recently, they extended CDT to handle binders by adopting Adam Chlipala's
idea of PHOAS \cite{PHOAS}, naming thier new extension as
\emph{parametric compositional data types} (PCDT).
In Section~3 of their paper on PCDT \cite{BahHvi12}, they give an enlightening
comparative summary on a series of studies on recursion schemes over
mixed-variant datatypes in the conventional setting ---
Meijer and Hutton \cite{MeiHut95}, Fegaras and Sheard \cite{FegShe96},
Washburn and Weirich \cite{bgb}, and their own.

PCDT is based on the conventional style, relying on ad-hoc polymorphism.
That is, they need to derive a class instance of an appropriate algebra
in order to define a desired recursion scheme for each datatype definition.
For example, a functor instance for iteration and a difucntor (or profunctor)
instance for iteration with inverses over regular datatypes.
Since conventional-style recursion schemes do not generalize naturally
to non-regular datatypes such as GADTs, they also need to derive different
class instances, that is, higher-order functor and difunctor instances for
non-regular datatypes. To alleviate this drawback of the conventional style,
they automate instance derivation by meta-programming using Template Haskell
for the PCDT library user.

On the contrary, the Mendler style, being based on higher-order
parametric polymorphism, enjoys uniform definitions of recursion schemes
across arbitrary kinds of datatypes, naturally generalizing from regular
to non-regular datatypes. In \S\ref{sec:ongoing:mphit}, we demonstrate
this elegance of the Mendler style by formulating a Mendler-style counterpart of
the conventional-style recursion scheme in PHOAS. Here, we summarize
the key idea how Bahr and Hvited \cite{BahHvi12} factored out
the fixpoint operator from recursive formulations of PHOAS,\footnote{
	An online posting of Edward Kmett \cite{PHOASforFree}
	also discusses PHOAS in a formulation very similar to PCDT.}
in order to lead up the discussion in \S\ref{sec:ongoing:mphit}.

In PCDT, they transfer the essence of PHOAS using two-level types
that are equipped with an extra parameter in base functors as well as
in the fixpoint operator. For example, their fixpoint operator and
the base functor for the untyped HOAS would be defined as:\footnote{
	Bahr and Hvited named their fixpoint operator \textit{Trm} in PCDT.
	Here, we call it $\hat\mu$ in order to use a notation similar to
	the other operators ($\mu$ and $\breve\mu$) in this paper.
	In addition, they compose base functors with multiple constructors
	such as \textit{ExpF} from several single constructor functors;
	hence, their library is named \emph{compositional}. Here, we focus
	the discussions on the \emph{parametric} flavor of their contribution.}
\begin{lstlisting}
data Mu_0 (f :: * -> * -> *) a = In_0 (f a (Mu_0 f a)) | Var_0 a
data ExpF r_ r = Lam (r_ -> r) | App r r
\end{lstlisting}
Their fixpoint operator \lstinline{Mu_0} takes a type constructor of kind
\lstinline{* -> * -> *} as an argument, unlike the previously discussed
fixpoint operators (\eg, \lstinline{Mu0} or \lstinline{Rec0}) that take
arguments of kind \lstinline{* -> *}. Note the use of two parameters
\lstinline{r_} and \lstinline{r} used in contravariant and covariant positions
respectively in the definition of \lstinline{ExpF}; the additional
parameter \lstinline{r_} is used in a contravariant recursive position
in the argument of the \lstinline{Lam} constructor.

Then, the recursive type for the untyped HOAS is defined as
the fixpoint of base \lstinline{ExpF}:
\begin{lstlisting}
type Exp' a = Mu_0 ExpF a   -- %\textcomment{pre-expressions that may contain $\textit{Var}_{*}$%}
type Exp = forall a. Exp' a   -- %\textcomment{\!$\textit{Var}_{*}$-free expressions enforced by parametricty%}
\end{lstlisting}
When \lstinline{ExpF}
is applied to \lstinline{Mu_0}, parameter \lstinline{r_} matches with
the answer type \lstinline{a}) and parameter \lstinline{r} matches with
the recursive type \lstinline{(Mu_0  ExpF  a)}.
Their \lstinline{Var_0} constructor of \lstinline{Mu_0} serves
the same purpose (\ie, injecting an inverse of an answer value)
as our \lstinline{Inverse0} constructor of \lstinline{Rec0}.

Finally, the constructor functions for the untyped HOAS are defined as follows:
\begin{lstlisting}
lam f = In_0 (Lam (f . Var_0)) -- :: (Mu_0 f a -> Mu_0 ExpF a) -> Mu_0 ExpF a
app e1 e2 = In_0 (App e1 e2)     -- :: Mu_0 ExpF a -> Mu_0 ExpF a -> Mu_0 ExpF a
\end{lstlisting}
Note the similarities between the types of the constructor functions above
and the types of the constructor functions in Listing~\ref{lst:HOASshow}.
A notable difference is where the inverse injection is used:
their \lstinline{Var_0} is used in the constructor function implementation
(\lstinline{lam}), while our \lstinline{Inverse0} is used in
the recursion scheme implementation (\msfit{*}).
