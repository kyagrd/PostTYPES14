\section{Summary and Future Work}

We reviewed Mendler-style iteration (\MIt{}) and
primitive recursion (\mpr{}) with their typical examples,
the list length function (Listing~\ref{lst:Len}) and
the factorial function (Listing~\ref{lst:mprim}), respectively.
\mpr{} extends \mit{} with the additional cast operation that
converts abstract recursive values to concrete recursive values.
Moreover, we reviewed Mendler-style iteration with syntactic
inverses (\msfit{}) with the HOAS formatting example (Listing~\ref{lst:HOASshow});
this is the ``hello world'' example of recursion schemes over
mixed-variant datatypes. The abstract inverse operation provided
by \msfit{}, which is not present in \MIt{}, makes it useful over
mixed-variant datatypes.

We formulated the type-preserving evaluator for the simply-typed HOAS
(Listing~\ref{lst:HOASeval}). This evaluator demonstrates the usefulness
of \msfit{} over indexed mixed-variant datatypes. Moreover, this
example is a novel theoretic discovery that type-preserving HOAS
evaluation can directly (\ie, without translation into intermediate
data structure such as first-order syntax) embedded into System \Fw\ 
because we proved termination of the HOAS evaluator by embedding
\msfit{} into System \Fw\ (\S\ref{sec:theory:embed}). Moreover,
we studied the equational properties of the embedding
(\S\ref{sec:theory:embed}-\ref{sec:theory:srpair}) and
the subtype relation between ordinary fixpoint types for \MIt{} and
their corresponding inverse-augmented fixpoint types for \msfit{}
(\S\ref{sec:murec}).

We introduced the idea of Mender-style iteration with a sized index
(\mprsi{}) motivated by the example of type-preserving evaluation
via semantic domain (Listing~\ref{lst:HOASevalV}), in contrast to 
the evaluation example via native values of the host language
using \msfit{} (Listing~\ref{lst:HOASeval}). \mprsi{} extends \mpr{}
with the additional abstract uncast operation, which is
the inverse of the abstract cast operation provided by \mpr{} as well.
However, the uncast operation needs to be restricted in order to
guarantee termination. Termination proof for \mprsi{} needs
further investigation.

We introduced Mendler-style iteration over PHOAS (\mphit{}) and
demonstrated its usefulness by writing the type-preserving evaluator
over typed PHOAS (Listing~\ref{lst:PHOASeval}); this is similar to
the HOAS evaluator using \msfit{} (Listing~\ref{lst:HOASeval}) but
even more succinct because abstract inverses are not needed.
Moreover, we can write examples using \mphit{} that are
not expressible using \msfit{} such as the desugaring of
higher-order syntax (Listing~\ref{lst:PHOASdesug}). We hope to show
termination of \mphit{} by finding its embedding in System \Fixw,
which is an extension of System \Fw\ that can embed \mpr{}.

Mendler-style recursion schemes naturally extends term-indexed datatypes
(\eg, length-indexed lists) so that one can express more fine-grained
properties of programs in their types Ahn, Sheard, Fiore, and Pitts
\cite{AhnSheFioPit13} developed a term-indexed calculi System $\F_i$
by extending System \Fw\ with term indices in order to embed
Mendler-style recursion schemes such as \MIt{} and \msfit{} over
term-indexed datatypes. System $\textsf{Fix}_i$ \cite{Ahn14thesis} is
a similar extension to System \Fixw\ that can embed \mpr{} and
(hopefully) \mphit{} over term-indexed datatypes.

Based on the theories of term-indexed calculi, we are designing and
implementing a language called Nax, named after \emph{Nax P. Mendler},
that supports Mendler-style recursion schemes over
both type- and term-indexed datatypes as native language constructs.
The Nax language \cite{Ahn14thesis} is designed to adopt advantages
of both functional programming languages (\eg, mixed-variant datatypes,
type inference) and dependently-typed proof assistants (\eg,
fine-grained properties, logical consistency). Semantics of Nax can be
understood by embedding its key constructs such as datatypes and
recursion schemes into the term-indexed calculi.

One of the challenges in the language design is to choose as many
useful set of Mendler-style recursion schemes, including ones
for mixed-variant datatypes, that have compatible embeddings
in a term-indexed calculus. Not all recursion schemes would necessarily
have close relationship between their fixpoint types, as the subtyping
relation between fixpoints of \MIt{} and \msfit{} discussed
in \S\ref{sec:murec}. \MIt{} and \mpr{} are compatible as well.
However, we think it may be difficult to find compatible embeddings for
both \mpr{} and \msfit{}. We hope to discover an embedding of \mphit{}
that is compatible with the embedding of \mpr{}.

There are several other features in consideration to develop Nax to
become a more powerful and practical language. Some have already been
implemented and awaiting theoretical clarifications, while others
are just preliminary thoughts:
restrictive form of kind polymorphsim, pattern match coverage checking,
generalization of arrow (\ie, function) types in abstract operations
to generalize Mendler-style recursion schemes even further (\eg,
monadic recursion \cite{bahr11wgp}), and handling computations that
cannot (or need not) be internally proved terminating by the type system
(\eg, bar types \cite{constable1987partial}, mobile types \cite{CasSjoWei14}).
